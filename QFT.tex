\documentclass[10pt]{article}

% Preamble

\usepackage{amsmath,amsfonts,amssymb}
\usepackage[mathscr]{euscript}
%\usepackage[mathcal]{euscript}
\usepackage{mathrsfs}
\usepackage{graphicx}
\usepackage{float}
\usepackage{bbm}
\usepackage{braket}
\usepackage{tikz-feynman}
\usepackage{simpler-wick}
\usepackage{cancel}
\usepackage{stackengine}
\usepackage{slashed}
\usepackage[colorlinks]{hyperref}
\hypersetup{linktocpage=true}

\newcommand{\bigzero}{\mbox{\normalfont\Large\bfseries 0}}

\title{Notes on Relativistic Quantum Field Theory: \\ A course given by Dr. Tobias Osborne}
\author{Transcribed by Alexander Victor St. John, Ph.D.}

% The Document

\begin{document}

\maketitle

\clearpage

{\hypersetup{linkcolor=blue}\tableofcontents} % Table of contents with clickable links.
\newpage

\section{Lecture 1: Motivations for (Relativistic) Quantum Field Theory}
\label{sec: lec1}

We assume that there is a "Theory of Everything", called ToE herein, which is suspected to include a contained description of small and large scale forces in the universe (e.g., strong nuclear, weak nuclear, electromagnetic, and gravitational forces). Our classical observations fall into a low energy, large scale, decoherent approximation of this ToE. The purpose of this collection of notes is to study \textit{(effective) relativistic quantum field theory} which is the low energy, large scale approximation of the ToE, and we are not sure how many "steps" there are in between the two, but the relativistic quantum field theory is the closest we currently come to a ToE. If we lower the energy and lengthen the scale of our study, we land in a nonrelativistic quantum field theory. A quantum field theory is then subject to decoherence, as we exist on the classical scale, and we begin the development of a relativistic quantum field theory with a classical field theory. In this sense, our purpose is to undo all of the approximations that nature forces us to take when we run experiments, take measurements, and devleop theoretical frameworks to describe observed phenomena.

\begin{figure}[H]
	\centering
	\includegraphics[width=\linewidth]{images/toe.png}
	\caption{Schematic of the study of field theories in physics.}
	\label{fig:fig1}
\end{figure}

\clearpage

\subsection*{Mathematical Machinery of Relativistic QFT}

For a quantum field theory to be relativistic it must be symmetric under the Poincar\'e group transformations.

\noindent Let the four-vector $(x_0, x_1, x_2, x_3)$ be the spacetime coordinates in an inertial reference frame. Then in any other reference frame that an observer chooses, the following condition is satisfied. 

\noindent (Note: We adopt the Einstein summation notation, and repeated indices are summed over, such that $\mu$, $\nu$, $\rho$, and $\sigma$ below are summed from $0$ to $3$.)

\begin{equation}
\eta_{\mu\nu} dx'^\mu dx'^\nu = \eta_{\rho\sigma} dx^\rho dx^\sigma
\end{equation}

\noindent Where we are using the Minkowski metric
\begin{equation}
\eta_{\mu\nu} 
= g_{\mu\nu} 
= \left( \begin{array}{rrrr} 
1 & 0 & 0 & 0 \\ 
0 & -1 & 0 & 0 \\ 
0 & 0 & -1 & 0 \\ 
0 & 0 & 0 & -1 
\end{array} \right)
.
\end{equation}

\noindent Any transformation satisfying the first equation, the relationship between the coordinates of two reference frames, must be a \textit{linear transformation} of the form, denoted by the pair $(\Lambda, a)$,
\begin{equation}
x'^\mu = \Lambda^\mu_{\,\,\,\nu} x^\nu + a^\mu
\end{equation}

\noindent Where $\Lambda^\mu_{\,\,\,\nu}$ is the $4x4$ Lorentz transformation matrix that represents rotations, and $a^\mu$ is a constant four-vector that represents spatial translations. 

\noindent Also note that the Lorentz transformation must satisfy the following condition, shown in index and matrix notation.
\begin{align}
	\eta_{\mu\nu} \Lambda^\mu_{\,\,\,\rho} \Lambda^\nu_{\,\,\,\sigma} &= \eta_{\rho\sigma} \\
	\Lambda^{T} \eta \Lambda &= \eta
\end{align}

\noindent These transformations $(\Lambda, a)$ are the elements of the Poincar\'e group $\mathcal{P}_4$. Let's check the group conditions, using equation \eqref{eq:eqn2} for calculating the product of two Poincar\'e transformations,

\begin{equation}
\begin{array}{llll} 
product & (\Lambda, a) \circ (\bar{\Lambda}, \bar{a}) & = & (\bar{\Lambda} \Lambda, \bar{\Lambda} a + \bar{a}) \\ 
identity & (1, 0) \circ (\Lambda, a) & = & (\Lambda, a) \\ 
inverse & (\Lambda^{-1}, a^{-1}) \circ (\Lambda, a) & = & (1, 0) \\ 
associativity & \left( (\Lambda, a) \circ (\bar{\Lambda}, \bar{a}) \right) \circ (\bar{\bar{\Lambda}}, \bar{\bar{a}}) & = & (\Lambda, a) \circ \left( (\bar{\Lambda}, \bar{a}) \circ (\bar{\bar{\Lambda}}, \bar{\bar{a}}) \right)
\end{array}
\end{equation}

\noindent In index notation, the total effect of the product of two Poincar\'e transfomations is written as
\begin{align}
x'^\mu &= \Lambda^\mu_{\,\,\,\nu} x^\nu + a^\mu \\
x''^\mu &= \bar{\Lambda}^\mu_{\,\,\,\rho} x'^\rho + \bar{a}^\mu
\end{align}

\noindent Quantum mechanical symmetries are representated by \textit{unitary} or \textit{anti-unitary} linear operators as elements of a separable Hilbert space $\mathscr{H}$ such that for each Poincar\'e transformation $(\Lambda, a) \in \mathcal{P}_4$, there exists a unitary transformation 
\begin{equation}
U(\Lambda, a) : \mathscr{H} \rightarrow \mathscr{H}
\end{equation}

\noindent The identity unitary transformation, and the product of any two unitary transformations of elements of the Poincar\'e group are physically indistinguishable from each other up to a phase factor $e^{i\phi}$, where $\phi \in \mathbb{R}$
\begin{align}
U(\bar{\Lambda}, \bar{a}) U(\Lambda, a) &= e^{i\phi((\Lambda, a),(\bar{\Lambda}, \bar{a}))} U(\bar{\Lambda} \Lambda, \bar{\Lambda} a + \bar{a}) \\
U(\mathbbm{1}, 0) &= e^{i\phi} \mathbbm{1} .
\end{align}

\noindent The family of transformations $U(\Lambda, a)$ that satisfy these equations is called the \textit{projective unitary representations of} $\mathcal{P}_4$, and is precisely what we need to establish a relativistic quantum field theory that includes translations, rotations, and Lorentz boosts as its transformations.\\

\noindent Consider the \textit{time translation subgroup} of the Poincar\'e group 
\begin{equation}
\{ (\Lambda=\mathbbm{1}, a=(t, 0, 0, 0)): t \in \mathbb{R} \} \subset \mathcal{P}_4
\end{equation}
\noindent And let $U$ be a unitary representation of $\mathcal{P}_4$. Then $V(t) = U(\mathbbm{1}, (t, 0, 0, 0))$ is a one-parameter family of unitary transformations, which is a group homomorphism, such that $V(s)V(t)=V(s+t)$, and is called a \textit{propagator}, and is a solution to the Schr\"{o}dinger equation, assuming the Hamiltonian $\hat{H}$ is self-adjoint and is stable, such that there at most a finite number of, preferrably zero, negative eigenvalues

\begin{equation}
\frac{d V(t)}{dt} = i \hat{H} V(t).
\end{equation}

\noindent So, having a unitary representation is equivalent to solving the time-dependent Schr\"{o}dinger equation, and we require that $U(\Lambda, a)$ is positive energy, such that the spectrum of eigenvalues of $\hat{H}$ is positive $spec(\hat{H}) \subset \mathbb{R}^+$, and stable. Otherwise, the system may be unstable and plunge to more and more negative eigenvalues. \\

\noindent All single-particle unitary representations of $\mathcal{P}_4$ have been classified by Wigner, and are labelled by their mass $m$ (invariant under Poincar\'e and Lorentz transformations) and their helicity/spin $s$. The bad news is that the universe is comprised of many particles, creating tension between locality and interactions, rendering thorough study of single-particle representations pointless. Even for a single particle, as we increase locality, self-interaction increases, and new particles are created via tunnelling. \\

\subsection*{Why is Obtaining a Relativistic Quantum Field Theory Hard?}

\noindent The reason that constructing a relativistic QFT is considered so difficult is because there are no nontrivial, only trivial, finite-dimensional unitary representations of the Poincar\'e group $\mathcal{P}_4$. The only nontrivial representations of $\mathcal{P}_4$ are infinite-dimensional, and are not easy to work with when constructing a unitary representation. Consider this in contrast to the, more easily constructed, special orthogonal group of three-dimensional rotations $SO(3)$ with unitary representations 

\begin{equation}
U = e^{i(\sigma_x \hat{J}_x + \sigma_y \hat{J}_y + \sigma_z \hat{J}_z)}. 
\end{equation}

\noindent Where $\hat{J}_\alpha$ are the angular momentum operators. The reason $SO(3)$ is easier to work with and construct a unitary representation is because $SO(3)$ a \textit{compact} group, while $\mathcal{P}_4$ is a \textit{non-compact} group.

\clearpage

\section{Lecture 2: Introduction to Classical Field Theory}
\label{sec: lec2}

\noindent To build a relativistic QFT, we start with an effective model from a \textit{classical field theory}, and make an "educated guess" to quantize the classical field theory. The desired relativistic QFT has nothing to do \textit{a priori} with the classical field theory. After quantization, the "educated guess", we take the limits of low energy, large scale, and decoherence, and check that we get back the classical field theory we started with, demonstrating whether the chosen quantization is correct or incorrect... enough, to some level of approximation.

\subsection*{Fields}

\noindent A \textit{field} is a quantity (e.g., density, spin, charge) tht is defined at every point on a manifold $\mathcal{M}$. Note that a rigorous definition of a field requires the introduction of vector bundles, of which we will not go so far. \\
We work on the Minkowski spacetime manifold $\mathcal{M} = \mathcal{M}_{1,3} = \mathbb{R}^1 \times \mathbb{R}^3$, with the field often taken to be two-times differentiable such that $\phi \in C^2(\mathcal{M}, \rho)$ and defined as a function from the manifold to some target space $\rho$
\begin{equation}
\phi : \mathcal{M} \rightarrow \rho 
\end{equation}

\noindent Some target spaces $\rho$, their associated field type, and example applications and models include

\begin{itemize}
\item $\rho = \mathbb{R}$; \textbf{scalar field}; charge density, magnetization density, Higgs boson
\item $\rho = \mathbb{R}^n$; \textbf{vector field}; electromagnetic field (actually a gauge field), pions
	\begin{figure}[H]
		\centering
		\includegraphics[scale=0.25]{images/vectorfield.png}
		\caption{Sketch of the vector field $\rho = \mathbb{R}^n$ over the Minkowski space $\mathcal{M}_{1,1}$.}
		\label{fig:fig1}
	\end{figure}
\item $\rho = \mathcal{S}^2$;  \textbf{vector field on the surface of a sphere};  $\sigma$-model, quantum magnets
\item $\rho = \mathcal{S}^1 \times \mathcal{S}^1$; \textbf{vector field on a torus}; Chern-Simons theory, Lie groups
\end{itemize}

\noindent Note that when our target space is the $N$-dimensional real vector field $\mathbb{R}^N$, the vector field is described as a list $N$ scalar fields $\{\phi_a(x)\}_{a=1}^N$, where $x$ is the coordinate four-vector.

\subsection*{Dynamics of Classical Fields}

\noindent We restrict this to discussion to classical dynamics generated by Lagrangians, obtained via the variational principle applied to th action functional, where the system of scalar fields $\{\phi_a(x)\}_{a=1}^N$, and $a$ labels the particle type (e.g., charge). The action functional $\mathcal{S}$ contains a function of the Langrangian density $\mathscr{L} = \mathscr{L}(\phi_a, \partial_\mu \phi_a)$. The Lagrangian density is actually a function of higher order derivatives of the fields $\phi_a$, but we make the assumption and approximation of first order derivatives, based on observation
\begin{equation}
\mathcal{S}(\Omega) = \int_\Omega d^4 x \,\, \mathscr{L}(\phi_a, \partial_\mu \phi_a). 
\end{equation}

\noindent Where $d^4 x = dx_0 dx_1 dx_2 dx_3$ and $\Omega \subset \mathcal{M}_{1,3}$, as a measurable set, is a region in $(3+1)$-dimensional spacetime. Typically, we consider the spacetime region as the entire Minkowski space $\Omega = \mathcal{M}_{1,3}$. \\

\noindent To extract the equations of motion, we suppose that the action $\mathcal{S}$ is stationary under infinitesimal variations of the component scalar fields $\phi_a(x) \rightarrow \phi_a(x) + \delta \phi_a(x)$, which vanish on the spacetime region boundary, such that $\delta\phi_a(x) = 0$ on $\partial\Omega$.  \\

\noindent Varying the action functional, we obtain $N$ \textit{Euler-Lagrange equations of motion}
\begin{align}
	\delta \mathcal{S} ( \Omega ) &= \int_\Omega d^4 x \,\, \left( \frac{\partial \mathscr{L}}{\partial \phi_a} \delta \phi_a + \frac{\partial \mathscr{L}}{\partial ( \partial_\mu \phi_a )} \delta ( \partial_\mu \phi_a ) \right) \\
	0 &= \int_\Omega d^4 x \,\, \left( \frac{\partial \mathscr{L}}{\partial \phi_a} \delta \phi_a - \frac{\partial}{\partial x^\mu} \left( \frac{\partial \mathscr{L}}{\partial ( \partial_\mu \phi_a )} \right) \right) \delta\phi_a + \int_\Omega d^4 x \,\, \frac{\partial}{\partial x^\mu} \left( \frac{\partial \mathscr{L}}{\partial ( \partial_\mu \phi_a )} \delta \phi_a \right) \\
	0 &= \int_\Omega d^4 x \,\, \left( \frac{\partial \mathscr{L}}{\partial \phi_a} \delta \phi_a - \frac{\partial}{\partial x^\mu} \left( \frac{\partial \mathscr{L}}{\partial ( \partial_\mu \phi_a )} \right) \right) \delta\phi_a + \partial_\mu \mathcal{M}^\mu.
\end{align}

\noindent The last term is a surface term which vanishes on the boundary, since we first demanded that $\delta\phi_a = 0$ vanishes on the boundary, such that $\partial_\mu \mathcal{M}^\mu = 0$ on $\partial\Omega$. \\

\noindent Since $\mathcal{S}(\Omega)$ is stationary with respect to all variations of the fields $\delta\phi_a$ and admissable spacetime regions $\Omega$, the integrand of the remaining term must also vanish $\forall \, a = 1, 2,... \, , N$, and we obtain the $N$ Euler-Lagrange equations of motion
\begin{equation}
-\frac{\partial \mathscr{L}}{\partial \phi_a} + \frac{\partial}{\partial x^\mu} \left( \frac{\partial \mathscr{L}}{\partial (\partial_\mu \phi_a)} \right) = 0.
\end{equation}

\noindent So, with equations of motion in hand, gotten by whatever means, they can be encoded in the Langrangian density, and then recovered via the Euler-Lagrange equations of motion. This allows us to discover equations of motion with certain properties, such as being symmetric under the Poincar\'e transformations, by designing Lagrangian densities, which are scalars under these symmetry transformations, apply Euler-Lagrange "recipe", and get the equations of motion, which are guaranteed to be symmetric under the chosen transformations. This benefit of the action principle makes it easy to design equations of motion with certain symmetries. \\

\subsection*{Example: Klein-Gordon field}

Consider the Langrangian density 
\begin{align}
\mathscr{L} &= \frac{1}{2}\dot{\phi}^2 - \frac{1}{2}(\nabla\phi)^2 - \frac{1}{2}m^2\phi^2 \\
&= \frac{1}{2}(\partial_\mu \phi)^2 - \frac{1}{2}m^2\phi^2
\end{align}

\noindent Where $\dot{\phi} = \partial_0 \phi$ is the time derivative of the field, and $(\nabla\phi)_j = \partial_j\phi$ for $j=1, 2, 3$ are the $x$, $y$, $z$ spatial components of the field derivatives. \\

\noindent Apply the Euler-Lagrange equations of motion to obtain the dynamics of the Klein-Gordon field
\begin{align}
-\frac{\partial \mathscr{L}}{\partial \phi_a} + \frac{\partial}{\partial x^\mu} \left( \frac{\partial \mathscr{L}}{\partial (\partial_\mu \phi_a)} \right) &= 0 \\
\frac{\partial}{\partial x^\mu}(\partial^\mu \phi) -(-m^2 \phi) &= 0 \\
\partial_\mu \partial^\mu \phi + m^2 \phi &=0 \\
\Box \phi + m^2 \phi &= 0
\end{align}

\subsection*{Hamiltonian formalism}

\noindent A better way to guess/build a quantum theory, with the correct classical limit determined by the Langrangian density, is the \textit{Hamiltonian formalism}, where we calculate conjugate variables and impose canonical (algebraic) commutation relations. \\

\noindent Suppose that $\phi_a(x)$ is a component field's \textit{canonical position}. Then define the \textit{conjugate momentum density} for each field to be 
\begin{equation}
\pi_a(x) = \frac{\partial \mathscr{L}}{\partial \dot{\phi_a}}.
\end{equation}

\noindent In order to computationally study a field, we discretize the space into a regular lattice with spacing $\epsilon$ between each lattice site, since representing a continuous object on a computer would require an infinite amount of data. Discretization yields generalized coordinates, defined by the field itself, of the form $q_j^a(t) = \phi_a(t, x_j), \, j \in \mathbb{Z}$. For example, consider the $1+1$-dimensional Minkowski space $\mathcal{M}_{1,1}$

\begin{figure}[H]
	\centering
	\includegraphics[width=\linewidth]{images/discrete.png}
	\caption{Schematic of field discretization.}
	\label{fig:fig3}
\end{figure}

\noindent To discretize the action functional and apply the variational principle, replace the partial derivatives $\partial_\mu \phi_a(t, x)$, since it is a continuous operation, by applying Taylor's theorem to approximate the spatial component of the field as a finite difference, where $\epsilon$ should be made as small as possible such that the error approximation is minimized
\begin{equation}
\partial_x \phi_a(t, x_j) \cong \frac{\phi_a(t, x_j + \epsilon) - \phi_a(t, x_j)}{\epsilon} = \frac{q^a_{j+1} - q^a_j}{\epsilon}.
\end{equation}

\noindent Leave the temporal component to be continuous
\begin{equation}
\partial_t \phi_a(t, x_j) \cong \dot{q}_j^a . 
\end{equation}

\noindent To obtain the Lagrangian, integrate the density over space, such that only the time dependence is left
\begin{equation}
\int d^3 x \,\, \mathscr{L} (\phi_a, \partial_\mu \phi_a) = L(q_j^a (t), \dot{q}_j^a (t)) = L(t) . 
\end{equation}

\noindent The discrete approximation of the Lagrangian is written as 
\begin{equation}
L(t) \cong \sum_j \delta x_j \mathscr{L}(\phi_a(t, x_j), \partial_\mu \phi_a(t, x_j)). 
\end{equation}

\noindent And the discrete conjugate momenta 
\begin{equation}
p_j^a(t) = \frac{\partial L}{\partial \dot{q}_j^a} = \sum_{j'} \delta {x_j'} \frac{\partial \mathscr{L}_{j'}}{\partial \dot{q}_j^a} = \delta x_j \pi^a(t, x_j).
\end{equation}

\subsubsection*{Simple Example}

\noindent As a simple example, consider the Lagrangian density 
\begin{align}
\mathscr{L} &= \frac{1}{2} (\partial_\mu \phi \partial^\mu \phi) \\
&= \frac{1}{2}(\partial_t \phi)^2 - \frac{1}{2} (\partial_x \phi)^2 .
\end{align}

\noindent (RECHECK) Integrate over space $d^3 x$ to obtain the Lagrangian, and discretize using the rules defined above
\begin{equation}
L(t) = \frac{1}{2} \sum_{j=-\infty}^{\infty} \left( \epsilon \left( \frac{d q_j}{d t} \right)^2 - \frac{(q_{j+1} - q_j)}{\epsilon} \right) .
\end{equation}

\noindent Where $\delta x_j = \epsilon$ and $p_j(t) = \epsilon \frac{d q_j}{dt} = \pi(t, x_j) \delta x_j $. 

\noindent As $\epsilon \rightarrow 0$, $q_j(t) \rightarrow \phi(t, x)$ and $p_j(t) \rightarrow \dot{\phi(t, x)}$

\subsubsection*{Hamiltonian Density}

\noindent For the discrete approximation, the Hamiltonian $H$, obtained by integrating the Hamiltonian density $\mathscr{H}$ of space $d^3 x$, is written as 
\begin{equation}
H = \sum_j p_j^a \dot{q}_j^a - L = \sum_j \delta x_j ( \pi_a(t, x_j) \dot{\phi}_a(t, x_j) - \mathcal{L}_j )
\end{equation}

\noindent In the limit as $\epsilon \rightarrow 0$, the Hamiltonian density is 
\begin{equation}
\mathscr{H}(t, x) = \pi_a(t, x) \dot{\phi}_a(t, x) - \mathscr{L} \left( \phi_a(t, x), \partial_\mu \phi_a(t,x) \right)
\end{equation}

\noindent The Hamiltonian density for the Klein-Gordon field is, dropping the field index and spacetime dependencies from the expression, 
\begin{equation}
\mathscr{H} = \frac{1}{2}\pi^2 + \frac{1}{2}(\nabla \phi)^2 + \frac{1}{2} m^2 \phi^2
\end{equation}


\clearpage

\section{Lecture 3: Symmetries in Classical Field Theory}
\label{sec: lec3}

\noindent Suppose that $\mathscr{L}(\phi_a, \partial_\mu \phi_a)$ is the Lagrangian density for some set of fields $\{ \phi_a(x) \}_{a=1}^N$. Recall that the Lagrangian density is a compact, encrypted of writing the equations of motion of a system, and the Euler-Lagrange equations and the principle of least action are used to unpack/decrypt the equations of motion. \\

\noindent Now consider an infinitesimal continuous transformation of the fields
\begin{equation}
\phi_a '(x) = \phi_a(x) + X_a(\phi_a)
\end{equation}

\noindent This produces an \textit{infinitesimal symmetry} in the equations of motion when they are left invariant under the principle of least action
\begin{align}
\mathscr{L} &\rightarrow \mathscr{L}(\phi_a', \partial_\mu \phi_a') = \mathscr{L}(\phi_a, \partial_\mu \phi_a) + \partial_\mu F^\mu \\
\& \,\,\,\, \mathcal{S} [ \phi_a ] &= \int d^4 x \mathscr{L} = \int d^4 x ( \mathscr{L} + \partial_\mu F^\mu )
\end{align}

\subsection*{Noether's Theorem}

\noindent Every continuous symmetry of a Lagrangian implies the existence of a conserved current $j^\mu(x)$. \\

\noindent \textbf{Proof:} Let $X_a[\phi_a] = \delta \phi_a$ be an arbitrary, infinitesimal change in each field, such that the infinitesimal change in the Lagrangian density is 

\begin{align}
\delta \mathscr{L}(\phi_a, \partial_\mu \phi_a) &= \mathscr{L}(\phi_a+\delta \phi_a, \partial_\mu(\phi_a + \delta \phi_a)) - \mathscr{L}(\phi_a, \delta_\mu \phi_a) \\
&\cong \frac{\partial \mathscr{L}}{\partial \phi_a} \delta \phi_a + \frac{\partial \mathscr{L}}{\partial (\partial_\mu \phi_a)} \delta(\partial_\mu \phi_a) \\
&= \left( \frac{\partial \mathscr{L}}{\partial \phi_a} - \partial_\mu \left( \frac{\partial \mathscr{L}}{\partial (\partial_\mu \phi_a)} \right) \right) \delta \phi_a + \partial_\mu \left( \frac{\partial \mathscr{L}}{\partial (\partial_\mu \phi_a)} \delta \phi_a \right) \\
\delta \mathscr{L}(\phi_a, \partial_\mu \phi_a) &= \partial_\mu \left( \frac{\partial \mathscr{L}}{\partial (\partial_\mu \phi_a)} \delta \phi_a \right)
\end{align}

\noindent Where in line $(14)$, we have kept only first order terms $\mathcal{O} (\delta \phi_a)$, and have used the Taylor expansion and added zero to get to line $(15)$. To obtain line $(16)$, note that the first term in line $(15)$ is equal to zero, since $\phi_a(x)$ obey the Euler-Lagrange equations. \\

\noindent As defined above, for an infinitesimal transformation, call $\delta \phi_a = X_a[\phi_a]$  and  for an infinitesimal symmetry, call $\delta \mathscr{L} = \partial_\mu F^\mu$.

\begin{align}
\partial_\mu F^\mu &= \partial_\mu \left( \frac{\partial \mathscr{L}}{\partial (\partial_\mu \phi_a)} X_a[\phi_a] \right) \\
0 &= \partial_\mu \left( \frac{\partial \mathscr{L}}{\partial (\partial_\mu \phi_a)} X_a[\phi_a] - F^\mu \right)
\end{align}

\noindent Call the conserved quantity the conserved current 
\begin{equation}
j^\mu (x) = \left( \frac{\partial \mathscr{L}}{\partial (\partial_\mu \phi_a)} X_a[\phi_a] - F^\mu \right).
\end{equation}

\noindent Now, a conserved current implies the existence of a conserved charge. For any measurable region in our Minkowski space $V \subset \mathcal{M}$, define the integral of the time-like component of the current as
\begin{equation}
Q_V = \int_V d^3 x \,\, j^0(x)
\end{equation}

\noindent Take $V=\mathbb{R}^3$, and assume that the current vanishes at infinity, such that $j \rightarrow 0$ on the boundary $\partial V$, and take the time derivative
\begin{align}
\frac{d Q_{\mathbb{R}^3}}{d t} &= \int_{\mathbb{R}^3} d^3x \,\, \partial_0 j^0 (x) \\
&= - \int_{\mathbb{R}^3} d^3x \,\, \partial_k j^k \\
&= - \int_{\partial \mathbb{R}^3} j  ds \\
\frac{d Q_{\mathbb{R}^3}}{d t} &= 0
\end{align}

\subsection*{Example}

\noindent Consider an active transformation of spacetime coordinates $x^\mu \rightarrow x^\mu - \epsilon^\mu$.

\noindent Then each field transforms as
\begin{align}
\phi'_a(x^\mu) &= \phi_a(x^\mu + \epsilon^\mu) \\
&= \phi_a(x^\mu) + \epsilon^\nu \partial_\nu \phi_a(x^\mu)
\end{align}

\noindent And the Lagrangian density transforms as, yielding $4 \times 4=16$ equations from summing over $\mu$ and $\nu$ 
\begin{align}
\mathscr{L}(x'^\mu) &= \mathscr{L} (x^\mu + \epsilon^\mu) \\
&= \mathscr{L}(x^\mu) + \epsilon^\nu \partial_\nu \mathscr{L}(x^\mu)
\end{align}

\noindent Where the infinitesimal field transformation is $X_a[\phi_a] = \epsilon^\nu \partial_\nu \phi_a(x^\mu)$, and the infinitesimal symmetry of the Lagrangian density is $\partial_\mu F^\mu = \epsilon^\mu \partial_\mu \mathscr{L}(x^\mu)$ \\

\noindent Consider the infinitesimal element $\epsilon$ with basis vector entries $[\hat{\nu}]^\mu_{\,\,\,\nu} = \delta^\mu_{\,\,\,\nu}$
\begin{equation}
\epsilon^\mu = \epsilon \hat{\nu}^\mu = \epsilon \{ (1,0,0,0), (0,1,0,0), (0,0,1,0), (0,0,0,1) \}.
\end{equation}

\noindent Apply Noether's theorem to each of the 4 symmetry terms $\epsilon \hat{\nu}^\mu$, yielding 16 total terms that we assign as the elements $T^\mu_{\,\,\,\nu}$ of the \textit{energy-momentum} or \textit{stress-energy tensor}
\begin{align}
T^\mu_{\,\,\,\nu} = j^\mu_{\,\,\,\nu} = \frac{\partial \mathscr{L}}{\partial (\partial_\mu \phi_a)} \partial_\nu \phi_a - \delta^\mu_{\,\,\,\nu} \mathscr{L}.
\end{align}

\noindent Note that each of the $\nu^{th}$ columns of the energy-momentum tensor correspond to one of the four conserved currents and translation in each of the $\nu^{th}$ directions
\begin{align}
\partial_\mu T^\mu_{\,\,\,\nu} = 0, \,\,\,\, \forall \,\, \nu.
\end{align}

\noindent In a \textit{closed} system, the corresponding conserved charges, from the columns (conserved currents) of the energy-momentum tensor, are the total energy and the momentum in each of the three spatial directions
\begin{align}
E &= \int d^3x \,\,\, T^{00} \\
p^j &= \int d^3x \,\,\, T^{0j} .
\end{align}

\subsubsection*{Example of the Example: Klein-Gordon Field}

Consider the Klein-Gordon Lagrangian density $\mathscr{L} = \frac{1}{2} \partial_\mu \phi \partial^\mu \phi - \frac{1}{2}m^2 \phi^2$. The energy-momentum tensor has elements of the form
\begin{align}
T^\mu_{\,\,\,\nu} &= \partial^\mu \phi \partial_\nu \phi - \delta^\mu_{\,\,\,\nu}\mathscr{L} \\
T^{\mu \nu} &= \eta^{\nu \nu'} T^\mu_{\,\,\,\nu'} = \partial^\mu \phi \partial^\nu \phi - \eta^{\mu \nu} \mathscr{L}.
\end{align}

\noindent The corresponding conserved charges are
\begin{align}
E &= \int d^3x \,\, \mathscr{H}(x) = \int d^3x \,\, (\partial^0 \phi \partial^0 \phi - \frac{1}{2}\partial^0 \phi \partial^0 \phi + \frac{1}{2} m^2 \phi^2) \\
p^j &= \int d^3x \,\, \dot{\phi} \partial^j \phi .
\end{align}

\subsubsection*{How To Apply Noether's Theorem}
\begin{enumerate}
\item Identify the continuous symmetry. \\
\item Calculate the change in the Lagrangian density. \\
\item Calculate the change in each of the fields. \\
\item Work out the conserved currents and charges.
\end{enumerate}

\subsection*{Infinitesimal Lorentz Transformations}

\noindent Consider the transformation $x^\mu \rightarrow \Lambda^\mu_{\,\,\,\nu} x^\nu$, where $\Lambda^\mu_{\,\,\,\nu} = \delta^\mu_{\,\,\,\nu} + \omega^\mu_{\,\,\,\nu}$, and $\omega^\mu_{\,\,\,\nu}$ is infinitesimal. Next, recall the following property of the group of Lorentz transformations, restricting the possible values for $\omega$
\begin{align}
\eta &= \Lambda^T \eta \Lambda \\
\eta^{\mu \nu} &= (\delta^\mu_{\,\,\,\sigma} + \omega^\mu_{\,\,\,\sigma} ) (\delta^\nu_{\,\,\,\tau} + \omega^\nu_{\,\,\,\tau} ) \eta^{\sigma \tau} \\
0 &=_{\mathcal{O}(\omega)} \omega^{\mu \nu} + \omega^{\nu \mu}
\end{align}

\noindent This is a linear equation in $\omega$, since we have kept only up to first order terms in $\omega$, and tells us that $\omega$ is an \textit{antisymmetric}, infinitesimal generator of Lorentz transformations with six independent variables which define six continuous symmetries and six conserved currents and charges.

\begin{equation}
\omega = \left(
\begin{array}{cccc}
0 & -\alpha & -\beta & -\gamma \\
\alpha & 0 & -\delta & -\epsilon \\
\beta & \delta & 0 & -\kappa \\
\gamma & \epsilon & \kappa & 0 \\
\end{array}
\right)
\end{equation}

\noindent The action of this infinitesimal Lorentz transformation on the fields is 
\begin{align}
\phi_a(x) \rightarrow \phi'_a(x) &= \phi_a(\Lambda^{-1}x) \\
&= \phi_a((\delta-\omega)x) \\
&= \phi_a(x^\mu - \omega^\mu_{\,\,\,\nu} x^\nu) \\ 
&=_{\mathcal{O}(\omega)} \phi_a(x) - \omega^\mu_{\,\,\,\nu} x^\nu \partial_\mu \phi_a(x)
\end{align}

\noindent Showing that the symmetry is defined by the infinitesimals
\begin{align}
\delta \phi_a &= -\omega^\mu_{\,\,\,\nu} x^\nu \partial_\mu \phi_a \\
\& \,\,\,\, \delta \mathscr{L} &= -\omega^\mu_{\,\,\,\nu} x^\nu \partial_\mu \mathscr{L} = -\partial_\mu(\omega^\mu_{\,\,\,\nu} x^\nu \mathscr{L}) \\
\& \,\,\,\, j^\mu_{\,\,\,\omega} &= \frac{\partial \mathscr{L}}{\partial (\partial_\mu \phi_a)} \omega^\rho_{\,\,\,\nu} x^\nu \partial_\rho \phi_a + \omega^\mu_{\,\,\,\nu} x^\nu \mathscr{L}.
\end{align}

\noindent (CHECK how $j$ to $\mathcal{J}$) Applying Noether's theorem tells us that the six independent conserved currents $\partial_\mu (\mathcal{J}^\mu)^{\rho \sigma} = 0$, and conserved charges, are of the form
\begin{align}
(\mathcal{J}^\mu)^{\rho \sigma} &= x^\rho T^{\mu \sigma} - x^\sigma T^{\mu \rho} \\ 
Q^{jk} &= \int d^3x \,\, (x^j T^{0k} - x^k T^{0j}) \\
Q^{0j} &= \int d^3x \,\, (x^0 T^{0j} - x^j T^{00})
\end{align}

\noindent Call $Q^{jk}$ the generators of rotations, and $Q^{0j}$ the generators of boosts of the Lorentz transformations.

\subsection*{Generators}

\noindent Let $f$ and $g$ be maps from phase space to the real numbers 
\begin{equation}
f, \, g: \,\, \mathbb{R}^N \times \mathbb{R}^N \rightarrow \mathbb{R}.
\end{equation}

\noindent Define the Poisson bracket with the pairs of canonical coordinates $(q_j, p_j)$
\begin{align}
\{f,g\} &= \sum_{j=1}^N \left( \frac{\partial f}{\partial q_j} \frac{\partial g}{\partial p_j} - \frac{\partial f}{\partial p_j} \frac{\partial g}{\partial q_j} \right) \\
\{f, H \} &= \frac{d f}{d t}
\end{align}

\noindent The field theory version of the Posson bracket is defined with the canonical coordinate pairs $(\phi(x), \pi(x))$
\begin{align}
\{F, G\} &= \int d^3x \,\, \left( \frac{\delta F}{\delta \phi(x)} \frac{\delta G}{\delta \pi(x)} - \frac{\delta F}{\delta \pi(x)} \frac{\delta G}{\delta \phi(x)} \right) \\
\{f, Q^{\rho \sigma}\} &= \frac{\partial f}{\partial s^{\rho \sigma}}
\end{align}

\noindent Where the Poisson bracket of $f$ and the conserved charges $Q$ generate the corresponding symmetry transformations. Conserved cahrges also obey the Lie algebra obeyed by the Poincar\'e group.

\subsection*{Standard Dogma of Quantization}

\noindent Basically, put hats on things

\begin{itemize}
\item Function $f$ on phase space $\rightarrow$ linear operator $\hat{f}$ (observable) on Hilbert space \\
\item Poisson bracket $\{ f,g \} = h \rightarrow$ commutator $[\hat{f}, \hat{g}] = i \hat{h}$ \\
\item Conserved charge $Q^{\rho \sigma} \rightarrow$ conserved charge operator $\hat{Q}^{\rho \sigma}$
\end{itemize}

\noindent Where the conserved charge operators generate the Lorentz transformations on Hilbert space
\begin{equation}
\frac{d \hat{U}}{d s} = i [\hat{U}, \hat{Q}^{\rho \sigma}] \omega_{\rho \sigma}
\end{equation}

\clearpage

\section{Lecture 4: Field Quantization}
\label{sec:lec4}

\noindent More than one quantum field theory can have the same classical field theory as an effective model, making field quantization not a well-posed problem. Developing a quantum field theory is therefore built on educated guesses.

\subsection*{Canonical quantization of particles} 
 
\noindent The standard approach of \textit{canonical quantization} is to begin with a classical theory and suppose $n$ classical degrees of freedom, which are used to measure the canonical coordinate pairs, \textit{position} $q_j$ and \textit{momentum} $p_j$, for each degree of freedom, such that the Poisson bracket is defined by $\{q_j, p_k \} = \delta_{jk}$. The total energy of the of the system is measured by the classical \textit{Hamiltonian}, defined by 
\begin{equation}
H = \sum_{j=1}^n \frac{p_j^2}{2 m} + \frac{m}{2} \sum_{j, k=1}^n q_j [\textbf{Q}]_{jk} q_k
\end{equation}

\noindent Where $\textbf{Q}$ is an $n \times n$ symmetric, positive matrix. \\

\noindent For example, consider the quantum harmonic oscillator, and take the naive approach by basically putting hats on everything. This ends up working for field quantization, and yields a unitary representation of the Poincar\'e group. \\

\begin{itemize}
\item \textbf{Canonical coordinates}: $(q_j, p_j) \\ \rightarrow \textbf{Canonical coordinate operators}: (\hat{q}_j, \hat{p}_j)$ \\
\item \textbf{Poisson bracket}: $\{ q_j, p_k \} = \delta_{jk}$ \\ $\rightarrow$ \textbf{Commutator}: $[\hat{q}_j, \hat{p}_k] = i \delta_{jk}$ \\
\item \textbf{Hamiltonian}: $H$, as defined above \\ $\rightarrow \textbf{Hamiltonian operator}: \hat{H} = \sum_{j=1}^n \frac{\hat{p}_j^2}{2 m} + \frac{m}{2} \sum_{j, k=1}^n \hat{q}_j [\textbf{Q}]_{jk} \hat{q}_k$
\end{itemize}

\noindent To diagonalize the Hamiltonian operator, first note that since $\textbf{Q}$ is a symmetric, positive $n \times n$ matrix, there exists an orthogonal matrix $\textbf{O}$ (s.t., $\textbf{O}^T \textbf{O} = \textbf{I}$), such that $\textbf{O} \textbf{Q} \textbf{O}^T = \textbf{D}$, where $\textbf{D}$ is a diagonal matrix, where we call the diagonal elements $\{ \omega_i^2 \}_{i=1}^n$ \\

\noindent Now, transform the canonical coordinates, using the orthogonal matrix, such that the correct commutation relation is still obeyed
\begin{align}
\hat{q}_j &= \sum_{k=1}^n [\textbf{O}]_{jk} \hat{q}'_k \\
\hat{p}_j &= \sum_{k=1}^n [\textbf{O}]_{jk} \hat{p}'_k \\
i \delta_{jk} &= [ \hat{q}'_j, \hat{p}'_k ]
\end{align}

\noindent Using the facts that $\textbf{O}^T = \textbf{O}^{-1}$ and $\textbf{O} \textbf{Q} \textbf{O}^T = \textbf{D}$, the Hamiltonian becomes diagonalized
\begin{align}
\hat{H} &= \sum_{j=1}^n \frac{\hat{p'}_j^2}{2 m} + \frac{m}{2} \sum_{j, k, l, m=1}^n \hat{q'}_l [\textbf{O}^T]_{jl} [\textbf{Q}]_{jk} [\textbf{O}^T]_{km} \hat{q'}_m \\
\hat{H} &= \sum_{j=1}^n \frac{\hat{p'}_j^2}{2 m} + \frac{1}{2} \sum_{k=1}^n \omega_k^2 \hat{q'}_k^2 \\
\hat{H} &= \frac{1}{2} \sum_{k=1}^n \omega_k (\hat{a}_k^\dagger \hat{a}_k + \frac{1}{2})
\end{align}

\noindent Where the annihilation and creation ladder operators that diagonalize the quantum harmonic oscillator Hamiltonian are defined as
\begin{align}
\hat{a}_k &= \sqrt{\frac{m \omega_k}{2}} (\hat{q}'_k + \frac{i}{m \omega_k} \hat{p}'_k) \\
\hat{a}_k^\dagger &= \sqrt{\frac{m \omega_k}{2}} (\hat{q}'_k - \frac{i}{m \omega_k} \hat{p}'_k)
\end{align}

\subsection*{Canonical Quantization of Fields}

\noindent To quantize the Klein-Gordon field, we follow the same seemingly naive approach of putting hats on everything. In this example fo quantizing a field, the continuous variable $x$ is used, in contrast to the discrete labels $j$ in the previous example of the quantum harmonic oscillator

\begin{itemize}
\item \textbf{Canonical coordinates}: $(\phi(x), \pi(x)) \\ \rightarrow \textbf{Canonical coordinate operators}: (\hat{\phi}(x), \hat{\pi}(x))$ \\
\item \textbf{Poisson bracket}: $\{ \phi(x), \pi(y) \} = \delta^{(3)}(x-y)$ \\ $\rightarrow$ \textbf{Commutator}: $[\hat{\phi}(x), \hat{\pi}(y)] = i \delta^{(3)}(x-y)$ \\
	\subitem Note that this is the \textit{equal time Poisson bracket}, such that $(x-y)$ is the spatial three-vector.
	\subitem Also note that this commutator is strange,  as it is comprised of "two self-adjoint operators and something's that not even a function" \\
\item \textbf{Hamiltonian}: $H_{KG} = \frac{1}{2} \int d^3x \,\, \left( \pi^2(x) + (\nabla \phi(x))^2 + \frac{1}{2} m^2 \phi^2(x) \right)$ \\ $\rightarrow \textbf{Hamiltonian operator}: \hat{H}_{KG} = \frac{1}{2} \int d^3x \,\, \left( \hat{\pi}^2(x) + (\nabla \hat{\phi}(x))^2 + \frac{1}{2} m^2 \hat{\phi}^2(x) \right)$
\end{itemize}

\noindent Essentially, replace discrete sums with continuous integrals, by switching to a continuous label $j \rightarrow x$ and a continuous dynamical variable $q_j \rightarrow q_x=\phi(x)$. Solving the quantum Hamiltonian, by analogy of the canonical quantization of particles, should be as simple as creating the analog of the $\textbf{Q}$ matrix and its diagonalization. \\

\noindent  Replacing sums by integrals allows the full diagonalization of $\textbf{Q}$, and, therefore, the full diagonalization of the Hamiltonian $\hat{H} = \hat{H}_{KG}$, but this does not yet yield a unitary representation of the Poincar\'e group or a valid relativistic quantum field theory. Diagonalizing the Hamiltonian only quantizes a one-parameter subgroup of the Poincar\'e group. The conserved currents, charges, and operators obeying the correct Lie algebra are still needed for a relativistic quantum field theory. \\

\subsubsection*{Diagonalization of the quantum field theory}

\noindent The diagonalization of a field theory begins with emergence of the \textit{Fourier transform}. Replace sums with integrals, and, since the matrix elements are described by two numbers, let's define a continuous function in two variables $K(x,y)$

\begin{equation}
\hat{q}_j = \sum_k [\textbf{O}]_{jk} \hat{q}'_k \,\,\,\, \rightarrow \,\,\,\, \hat{\phi}(x) = \int d^3 y \,\, K(x, y) \hat{\phi}(y)
\end{equation}

\noindent Where $K(x,y)$ is the \textit{kernel} of the Fourier transform. \\

\noindent Using the Fourier transform is motivated by certain features of symmetric matrices. Consider the \textit{circulant} matrix, a type of Toeplitz matrix where each successive column is a cyclic permutation of the previous column, initialized by the first column vector, and has the form as an $n \times n$ matrix 

\begin{equation}
\begin{bmatrix} 
 c_0 & c_{n-1} & \dots & c_2 & c_1 \\
 c_1 & c_0 & c_{n-1} &  & c_2 \\
 \vdots & c_1 & c_0 &  \ddots & \vdots \\
 c_{n-2} &  & \ddots &  \ddots & c_{n-1} \\
 c_{n-1} & c_{n-2} & \dots &  c_1 & c_0 
\end{bmatrix}.
\end{equation} \\

\noindent These matrices are diagonalized via the discrete Fourier transform, which is an $n \times n$ unitary matrix, though not orthogonal and may have complex entries

\begin{equation}
\textbf{U} = \frac{1}{\sqrt{n}} 
\begin{bmatrix} 
 1 & 1 & 1 & \dots &  \\
 1 & \mu & \mu^2 & \dots &  \\
 1 & \mu^2 & \mu^4 &   & \vdots \\
 \vdots & \vdots & &  \mu^{jk} &  \\
  &  & \dots &  & \mu^{nn} 
\end{bmatrix}
\end{equation}


\noindent Where $\mu = e^{\frac{2 \pi i}{n}}$ is the $n^{th}$ roots of unity. The elements of the discrete Fourier transform, therefore, have the form $\frac{1}{\sqrt{n}} e^{\frac{2\pi i j k}{n}}$. Compare this to the continuous Fourier transform kernel function $K(x,y) = \frac{1}{2\pi} e^{ixy}$. \\

\noindent For transformations between position and momentum space, make the guess that the Fourier transform that will diagonalize our quantized Klein-Gordon Hamiltonian has the form 

\begin{equation}
\hat{\phi}(x) = \int \frac{d^3 p}{(2\pi)^3} e^{i p \cdot x} \hat{\phi}_p(p).
\end{equation}

\noindent Where $\hat{\phi}_p(p)$ is the momentum space wavefunction, and is not Hermitian, such that $\hat{\phi}_p(p)^\dagger = \hat{\phi}_p(-p)$. To check if the guess is correct, apply $\hat{H}_{KG}$ to the transform defined above, and observe whether it is diagonalized or not. \\

\noindent As in the discrete case of diagonalization, we construct ladder operators

\begin{align}
\hat{\phi}(x) &= \int \frac{d^3 p}{(2 \pi)^3} \frac{1}{\sqrt{2 \omega_p}} \left( \hat{a}_p e^{ip \cdot x} + \hat{a}_p^\dagger e^{-ip \cdot x} \right) \\
\hat{\pi}(x) &=-i  \int \frac{d^3 p}{(2 \pi)^3} \sqrt{\frac{\omega_p}{2}} \left( \hat{a}_p e^{ip \cdot x} - \hat{a}_p^\dagger e^{-ip \cdot x} \right) \\
&\omega_p = \sqrt{ |p|^2 + m^2}
\end{align}

\noindent Check the commutation relation
\begin{align}
[ \hat{\phi}(x), \hat{\pi}(x') ] &= \frac{-i}{2} \int \frac{d^3 p d^3 p'}{(2 \pi)^6} \sqrt{\frac{\omega_{p'}}{\omega_p}} \left( [\hat{a}_{-p}^\dagger, \hat{a}_{p'}] - [\hat{a}_p, \hat{a}_{-p'}^\dagger] \right) e^{i(p \cdot x+p' \cdot y)} \\
& \,\,\,\, \,\,\,\, \,\,\,\, \,\,\,\, \,\,\,\, \left( [\hat{a}_p, \hat{a}_{p'}^\dagger] = (2\pi)^3 \delta^{(3)}(p-p') \cdot \mathbb{I} \right) \\
 &= i \delta^{(3)}(x-y) 
\end{align}

\noindent Making this substitution, the quantum Klein-Gordon Hamiltonian is diagonalized

\begin{align}
\hat{H}_{KG} &= \int d^3 x \int \frac{d^3 p d^3 p'}{(2 \pi)^6} e^{i(p+p') \cdot x} ( \frac{-1}{4} \sqrt{\omega_p \omega_{p'}} (\hat{a}_p - \hat{a}_{-p}^\dagger)(\hat{a}_{p'} - \hat{a}_{-p'}^\dagger) \\
&\,\,\,\, \,\,\,\, + \frac{-pp' + m^2}{4 \sqrt{\omega_p \omega_{p'}}} (\hat{a}_p + \hat{a}_{-p}^\dagger)(\hat{a}_{p'} + \hat{a}_{-p'}^\dagger) ) \\
&= \int \frac{d^3 p}{(2\pi)^3} \omega_p (\hat{a}_p^\dagger \hat{a}_p + \frac{1}{2} [\hat{a}_p, \hat{a}_p^\dagger]) \\
&= \int \frac{d^3 p}{(2\pi)^3} \omega_p (\hat{a}_p^\dagger \hat{a}_p + \frac{1}{2} \delta_{pp} \cdot \mathbb{I}) \\
&\cong \int \frac{d^3 p}{(2\pi)^3} \omega_p \hat{a}_p^\dagger \hat{a}_p
\end{align}

\noindent Where the infinite absolute energy shift is tossed to get the last line, since we only measure energy differences, and $\hat{H}_{KG}$ is diagonalized!

\clearpage

\section{Lecture 5: Scalar Quantum Field Theory}
\label{sec:lec5}

\noindent By analogy to the classical Klein-Gordon equation and Hamiltonian, a model for the (equal time, $t=0$) quantum Klein-Gordon Hamiltonian was constructed and diagonalized, via (continuous) Fourier transform and ladder operators, 
\begin{align}
\hat{H}_{KG} &= \frac{1}{2} \int d^ 3 x \,\, \hat{\pi}^2(x) + (\nabla \hat{\phi}(x))^2 + m^2 \hat{\phi}^2(x) \\
\hat{H}_{KG} &= \int \frac{d^3 p}{(2\pi)^3} \,\, \omega_p \hat{a}_p^\dagger \hat{a}_p
\end{align}

\noindent Where the zeroth, time, component of the momentum 4-vector $\textbf{p}_0 = \omega_p = \sqrt{|p|^2 + m^2}$ depends on the spatial 3-vector $p$ and the constant $m^2$. \\

\noindent This yields a representation of a one parameter subgroup of the Poincar\'e group, namely $U((t,0,0,0))=e^{-it\hat{H}_{KG}}$, but a true relativistic quantum field theory requires the full (projective) unitary representation of the Poincar\'e group, including generators for all possible transformation: 10 Lorentz + 4 translation = 14 total transformations in the Poincar\'e group. \\

\noindent To quantize, put hats on the conserved charges identified by Noether's theorem: $Q_\alpha \rightarrow \hat{Q}_\alpha$. First, consider the generators of spatial translations, namely momentum. Recall that the classical conserved current $T^{0j}$ gives these, which is quantized: $p^j \rightarrow \hat{p}^j$.

\noindent Recall the classical energy-momentum tensor for the Klein-Gordon field
\begin{equation}
T^\mu_{\,\,\,\, \nu} |_{KG} = \partial^\mu \phi \partial_\nu \phi - \delta^\mu_{\,\,\,\, \nu} (\partial^\mu \phi \partial_\mu \phi - \frac{1}{2} m^2 \phi^2)
\end{equation}

\noindent From this, quantize and calculate the conserved charge for temporal translations, namely the Hamiltonian, and conserved charges for spatial translations, namely linear momentum.

\begin{align}
\hat{H}_{KG} &= \int d^3 x \,\, \hat{T}^{00} = \int \frac{d^3 p}{(2\pi)^3} \omega_p \hat{a}_p^\dagger \hat{a}_p\\
\hat{p}^j &= \int d^3 x \,\, \hat{T}^{0j} = \int d^3 x \,\, \hat{\dot{\phi}} \partial_j \hat{\phi} = \int d^3 x \,\, \hat{\pi} \partial_j \hat{\phi} \hat{p}^j = \int \frac{d^3 p}{(2\pi)^3} \,\, p^j \hat{a}_p^\dagger \hat{a}_p
\end{align}

\noindent Note that there are several choices for the ordering of $\hat{\pi}$ and $\hat{\phi}$ in the expression of $\hat{p}^j$ matters, and here is written the one that works. \\

\noindent Now check that the four-vector obeys the commuation relations, using the diagonalized momenta $\hat{p}^j = \int \frac{d^3 p}{(2\pi)^3} \,\, p^j \hat{a}_p^\dagger \hat{a}_p$
\begin{align}
\{Q_\alpha, Q_\beta\}_{PB} = f_{\alpha \beta}^{\,\,\,\, \gamma} Q_\gamma &\rightarrow [\hat{Q}_\alpha, \hat{Q}_\beta] = i f_{\alpha \beta}^{\,\,\,\, \gamma} Q_\gamma \\
\{p^\mu, p^\nu\}_{PB} = 0 &\rightarrow [\hat{p}^\mu, \hat{p}^\nu] = 0
\end{align}

\noindent This confirms a projective unitary representation of the \textit{translation subgroup} of the Poincar\'e group, and now construct the explicit Hilbert space as a Fock space, since the operators are quantized and diagonalized via ladder operators. \\

\noindent To construct a Fock space, begin by defining the vacuum state, highest weight vector in the language of representation theory,  $\ket{\Omega}$ such that the annihilation operator will completely obliterate it: $\hat{a}_{\textbf{p}} \ket{\Omega} = 0, \,\, \forall \, p$, where $\textbf{p}=(\omega_p, p)$. \\

\noindent The Hilbert space would then be generated via all finite linear combinations of vectors of the form 
$\ket{\textbf{p}_1 \textbf{p}_2 \dots \textbf{p}_n} = \hat{a}_{\textbf{p}_1}^\dagger \hat{a}_{\textbf{p}_2}^\dagger \dots \hat{a}_{\textbf{p}_n}^\dagger \ket{\Omega}$, but there is a technical issue of the $n$-dimensional momentum state vectors actually being improper vectors that are not normalizable, such that the scalar product needed to finish the defintion of the Hilbert space will always blow up to infinity, since $\braket{p|q} = (2 \pi)^3 \delta^{(3)}(p-q)$. These states are also not preparable by experiment, since the state vector $\ket{\textbf{p}_1 \, \textbf{p}_2 \dots \textbf{p}_n}$ represents $n$ delta functions in position-momentum space.  \\

\noindent To create a normalizable state that can be used to define the Hilbert space, "smear out" the momentum states by defining a smooth ($L^2$) function $\psi$, which must be Lorentz invariant, though the invariance it is not obvious
\begin{equation}
\ket{\psi} = \int \frac{d^3 p}{(2\pi)^3} \psi(\textbf{p}) \ket{\textbf{p}} = \int \frac{d^3 p}{(2\pi)^3} \psi(\textbf{p}) \hat{a}_p^\dagger \ket{\Omega}.
\end{equation}

\noindent Now, introduce a method to normalize these improper vectors to a new set of improper vectors that are manifestly, more obviously, Lorentz invariant, and offer a nice parameterization to make many calculations easier. \\

\noindent Consider the projection operator onto a single particle state, and note that the integrand and the integral (volume element) are both separately not invariant

\begin{equation}
\mathbb{I}_{single} = \int \frac{d^3 p}{(2\pi)^3} \ket{p}\bra{p}.
\end{equation}

\noindent Enter a reference frame where this state is invariant by multiplying by one

\begin{equation}
\mathbb{I}_{single} = \int \frac{d^3 p}{(2\pi)^3 X(\textbf{p})} X(\textbf{p}) \ket{p}\bra{p}.
\end{equation}

\noindent Where $X(\textbf{p})$ is a mystery factor to make the integral and integrand invariant. \\

\noindent \textbf{Claim:} $X(\textbf{p}) = \frac{2 \omega_p}{(2\pi)^3}$, where $p$ here must be the momentum 4-vector, since we are using the zeroth, or time, component $p_0 = \omega_p = \sqrt{|p|^2 + m^2}$. \\

\noindent \textbf{Proof:} \\

\noindent First, observe that $\int d^3 p$ is not Poincar\'e invariant, but $\int d^4 p$ is, such that $\int d^4 p = \int d^4 p'$, where $p'^\mu = \Lambda^\mu_{\,\,\nu} p^\nu + a^\mu$ is a Poincar\'e transformation, and $\Lambda^\mu_{\,\,\nu}$ is the Jacobian of the transformation, and $det(\Lambda)= \pm 1$, $\forall \, \Lambda$ unitary transformation. \\

\noindent Now notice that $p^\mu p_\mu = const. = m^2$ (4-vector length invariant), whose solution is the dispersion relation for a single relativistic particle, and has two branches $\textbf{p}_0 = \omega_{p} = \pm \sqrt{|p|^2 + m^2}$, where $|p|$ is the norm of the momentum (spatial) 3-vector.  \\

\noindent Restrict to the positive upper branch, and consider the Poincar\'e invariant quantity

\begin{equation}
\int d^4 p \,\, \delta(\textbf{p}_0^2 - |p|^2 - m^2)|_{\textbf{p}_0>0} = \int \frac{d^3 p}{2\textbf{p}_0}|_{\textbf{p}_0=\omega_{p}}.
\end{equation}

\noindent Therefore, to make the single particle state projection operator from above Poncar\'e invariant, compare terms in the line above to the "mystery factor" expression, proving that $X(\textbf{p}) = \frac{2 \omega_p}{(2\pi)^3}$. \\

\noindent Thus, the "delta normalization" of 3-vectors is defined via 

\begin{equation}
2 \omega_{p} \delta^{(3)} (p-q).
\end{equation}

\noindent And the renormalized, Lorentz invariant momentum 4-vector is built as 

\begin{equation}
\ket{\textbf{p}} = \sqrt{2 \omega_{p}} \ket{p} = \sqrt{2 \omega_{p}} \hat{a}_{p}^\dagger \ket{\Omega}.
\end{equation}

\noindent And the Lorentz invariant four-length comes out to be

\begin{equation}
\braket{\textbf{p}|\textbf{q}} = 2 (2 \pi)^3 \omega_{p} \delta^{(3)} (p - q). 
\end{equation}

\noindent Now, to express the operators in terms of the Fock vector space we build on top of the vectors $\ket{\textbf{p}}$, and determine the action of the generator of spacetime translations, the 4-momentum operator, $\hat{p}^\mu$ on the Hilbert space of momentum states (improper vectors) $\ket{\textbf{p}_1 \, \textbf{p}_2 \dots \textbf{p}_n}$. \\

\noindent This requires some commutation relations with the ladder operator $\hat{a}_{\textbf{p}}$ in the following lemma. \\

\noindent \textbf{Lemma:} $[ \hat{H}_{KG}, \hat{a}_p ] = - \omega_{p} \hat{a}_{p}$ and $[ \hat{p}^j, \hat{a}_\textbf{p} ] = p^j \hat{a}_p $. \\

\noindent Next follows the corollary, demonstrating that the operator $\hat{p}^\mu$ is \textit{diagonalized} in this Hilbert space basis, such that the 4-momentum operator annihilates the vacuum state: $\hat{p}^\mu \ket{\Omega} = 0$. \\

\noindent \textbf{Corollary:} $\hat{p}^\mu \ket{\textbf{p}_1 \textbf{p}_2 \dots \textbf{p}_n} = ( \sum_{j=1}^n p_j^{\, \mu} ) \ket{\textbf{p}_1 \textbf{p}_2 \dots \textbf{p}_n} $. \\

\subsection*{Lorentz Invariance in the Heisenberg Picture}

\noindent So, this operator allows unitary quantum spacetime translations, in the Schr\"odinger picture, via the exponentiated Hermitian operator quantity $U(a) = e^{-i a_\mu \hat{p}^\mu}$. \\

\noindent Now, to manifest any symmetries that may have not been shown in the Schr\"odinger picture, explore Lorentz invariance in the Heisenberg picture, which is also later helpful in perturbation theory. Real space calculations, at a specific spacetime location (e.g., $(t,\textbf{x})$) are also much easier in the Heisenberg picture than in the "spread-out" Fourier transformed Schr\"odinger picture.\\

\noindent To enter the Heisenberg picture, where time is explicitly included, an operator $\mathcal{O}$ is unitarily transformed, and its time evolution is determined via the Hamiltonian in the Heisenberg equation of motion
\begin{align}
\mathcal{O}_H &= e^{i \hat{H} t} \mathcal{O} e^{-i \hat{H} t} \\
\frac{d \mathcal{O}_H}{d t} &= i [ \hat{H}, \mathcal{O}_H ].
\end{align}

\noindent In the Heisenberg picture, the commutation relations for the canonical position and momentum operators become
\begin{align}
[ \hat{\phi}_H(t, x), \hat{\phi}_H(t, y) ] &= [ \hat{\pi}_H(t, x), \hat{\pi}_H(t, y) ] = 0 \\
[ \hat{\phi}_H(t, x), \hat{\pi}_H(t, y) ] &= i \delta^{(3)}(x - y).
\end{align}

\noindent Evolve the canonical position and momentum operators in time via the (spatially localized) Heisenberg equation of motion
\begin{align}
\frac{d \hat{\phi}(t, x)}{d t} &= i [ \hat{H}_{KG}, \hat{\phi}(t, x)] = \hat{\pi}(t, x) \\
\frac{d \hat{\pi}(t, x)}{d t} &= i [ \hat{H}_{KG}, \hat{\pi}(t, x)] = \nabla^2 \hat{\phi}(t, x) + m^2 \hat{\phi}(t, x).
\end{align}

\noindent Where the second equality is gotten by using integration-by-parts. Substitute the first equality for $\hat{\pi}(t, x)$ into the second equality, and combine the second derivatives of space and time, to show that the canonical field position operator obeys the Klein-Gordon equation
\begin{equation}
(\partial^\mu \partial_\mu + m^2 ) \hat{\phi}(t, x) = 0.
\end{equation}

\noindent This completes the development of the unitary representation of spacetime translations. Rotations and boosts are yet to be integrated into the unitary representation of the Poincar\'e group.

\clearpage

\section{Lecture 6: Causality in Scalar QFT}
\label{sec:lec6}

\noindent Thus far, we have diagonalized the Klein-Gordon Hamiltonian
\begin{equation}
\hat{H}_{KG} = \frac{1}{2} \int d^3 x \left( \hat{\pi}^2 (x) + (\nabla \hat{\phi}(x))^2 + m^2 \hat{\phi}^2(x) \right) = \int \frac{d^3 p}{(2\pi)^3} \hat{a}_p^\dagger \hat{a}_p
\end{equation}

\noindent And used this to construct the unitary operator that allows us to study the dynamics of Klein-Gordon's solution to Schr\"odinger's equation in the Heisenberg picture
\begin{equation}
\hat{U} = e^{-i\hat{H}_{KG} t}.
\end{equation}

\noindent We found the spatial solutions $\hat{\phi}(x)$, in the Schr\"odinger picture, for the Klein-Gordon equation, to be the position field operators
\begin{equation}
\hat{\phi} (x) = \int \frac{d^3 p}{(2\pi)^3} \frac{1}{\sqrt{2 \omega_p}} \left( \hat{a}_p e^{i p \cdot x} + \hat{a}_p^\dagger e^{-i p \cdot x} \right)
\end{equation}

\noindent \textbf{Notation:} The spatial 3-vectors of position $x$ and momentum $p$ are no longer bold-faced, and the spacetime 4-vectors will be bold-faced, such that 
\begin{align}
\textbf{x} &= (x^0, x) &= (t,x) &= (x^0, x^1, x^2, x^3) \\
\textbf{p} &= (p_0, p) &= (\omega_p, p) &= (x^0, x^1, x^2, x^3) .
\end{align}

\noindent These solutions are represented in the Heisenberg picture via
\begin{equation}
\hat{\phi}(t, x) = \hat{U}^\dagger \hat{\phi}(x) \hat{U} = e^{i \hat{H}_{KG} t} \hat{\phi}(x) e^{-i \hat{H}_{KG} t}.
\end{equation}

\noindent We still need to complete the (projective) unitary representation of the Poincar\'e group, including translations, boosts, and rotations, since we are doing \textit{relativistic} quantum field theory.  \\

\noindent The next step here is to check that $\hat{\phi}(t,x)$ respects causality, such that if two spacetime events are space-like separated, then they have no influence on each other. Note that if the two spacetime events are time-like, there may be influence. \\

\noindent Recall the commutation relation of the Hamiltonian (dropping subscript "$KG$") and the ladder operator
\begin{equation}
[ \hat{H}, \hat{a}_p ] = -\omega_p \hat{a}_p \implies e^{i \hat{H} t} \hat{a}_p e^{-i \hat{H} t} = e^{-i\omega_p t} \hat{a}_p.
\end{equation} 

\noindent Substituting $\hat{\phi}(x)$ into the Heisenberg picture, and using the above commutation relation, we have the field oeprator in the Heisenberg picture
\begin{equation}
\hat{\phi}(t, x) = \int \frac{d^3 p}{(2\pi)^3} \frac{1}{\sqrt{2 \omega_p}} \left( \hat{a}_p e^{i \textbf{p} \cdot \textbf{x}} + \hat{a}_p^\dagger e^{-i \textbf{p} \cdot \textbf{x}} \right)
\end{equation}

\noindent Consider the delocalized ("smeared out") field operator $\hat{\phi}(t,x)$ as an observable that samples the field at a localized spacetime location $\textbf{x}=(t,x)$. The question is whether this interpretation respects causality. \\

\noindent Consider a (projective) measurement event at $(t,x)$ of the quantum field $\hat{\phi}(t,x)$. This disturbance of the field should fly out at the speed of light along its spacetime light cone. The fact is that measuring the field causes an instantaneous disturbance \textit{everywhere}, but relativity is safe since we can not signal, send information, faster than the speed of light (outside of the forward light cone of the measurement event). \\

\begin{figure}[H]
	\centering
	\includegraphics[scale=0.3]{images/lightcone.png}
	\caption{Sketch of expected propagation of information (at the speed of light) due to measurement event of the field $\hat{\phi}(t,x)$.}
\end{figure}

\noindent The result is that no information may be transmitted via the field across a space-like interval ($(x_A - x_B)^2 < 0$). \\

\noindent Now, we have to agree on which quantities are observable, and a natural guess is to study the correlation function
\begin{equation}
\bra{0} \hat{\phi}(x^0,x) \hat{\phi}(y^0,y) \ket{0}.
\end{equation}

\noindent This is unfortunately wrong, since the correlation function has no operational meaning, and can not be directly measured, since the field operators are not Hermitian, in general. \\

\noindent \textbf{Digression: Interference experiment} \\

\noindent To study correlation functions which can be measured in the lab via experiment, consider the following interference experiment set up with a Klein-Gordon field with auxiliary modes of light used to perform measurements and generate results. \\

\begin{figure}[H]
	\centering
	\includegraphics[scale=0.4]{images/interference.png}
	\caption{Schematic of experiment to measure the correlation of two spacetime locations and the Klein-Gordon field under measurement.}
\end{figure}

\begin{enumerate}
\item Prepare the field and auxiliary system (left and right) in the vacuum state
	\begin{equation}
	\ket{0}_{field} \ket{0}_{left} \ket{0}_{right}.
	\end{equation}
\item Apply the Hadamard gate (beam splitter) to the left and right auxiliary states
	\begin{equation}
	\ket{0}_{field} \left( \frac{1}{\sqrt{2}} \left( \ket{0}_{left} \ket{1}_{right} + \ket{1}_{left} \ket{0}_{right} \right) \right).
	\end{equation}
\item Create a particle at $\textbf{x}$ or $\textbf{y}$ via the unitary operator
	\begin{equation}
	\hat{U}(\textbf{x}) \otimes \ket{1}\bra{1}_{left} \otimes \mathbb{I}_{right} + \hat{U}(\textbf{y}) \otimes \mathbb{I}_{left} \ket{1}\bra{1}_{right}
	\end{equation}
	Applied to the state in \textbf{Step 2}, which evolves to the state (dropping "left" and "right" labels)
	\begin{equation}
	\frac{1}{\sqrt{2}} \left( \hat{U}(\textbf{y}) \ket{0}_{field} \ket{0 \, 1} + \hat{U}(\textbf{x}) \ket{0}_{field} \ket{1 \, 0} \right)
	\end{equation}
\item Apply a second beam splitter to the auxiliary states to check for interference in the final state
	\begin{equation}
	\frac{1}{2} \left( \hat{U}(\textbf{x}) + \hat{U}(\textbf{y}) \right) \ket{0}_{field} \ket{0 \, 1} + \frac{1}{2} \left( \hat{U}(\textbf{x}) - \hat{U}(\textbf{y}) \right) \ket{0}_{field} \ket{1 \, 0}
	\end{equation}
\item Detect auxiliary states $\ket{0 \, 1}$ and $\ket{1 \, 0}$.  
\end{enumerate}

\noindent The probability of measuring one of the states, $\ket{0 \, 1}$, for example, is 
\begin{align}
\mathcal{P}(\ket{0 \, 1}) &= \frac{1}{4} \bra{0}_{field} \left( \hat{U}^\dagger(\textbf{x}) + \hat{U}^\dagger(\textbf{y}) \right) \left( \hat{U}(\textbf{x}) + \hat{U}(\textbf{y}) \right)  \ket{0}_{field} \\
&= \frac{1}{2} + \frac{1}{2} Re \left[ \bra{0}_{field} \hat{U}^\dagger(\textbf{x}) \hat{U}(\textbf{y}) \ket{0}_{field} \right] \\
&= \frac{1}{2} + \frac{1}{2} Re \left[ \bra{0}_{field} e^{-i \epsilon \hat{\phi}(\textbf{x})} e^{i \epsilon \hat{\phi}(\textbf{y})} \ket{0}_{field} \right] \\
\mathcal{P}(\ket{0 \, 1}) &= \frac{1}{2} + \frac{1}{2} Re \left[ \bra{0}_{field} e^{-i \epsilon \hat{\phi}(\textbf{x})+ i \epsilon \hat{\phi}(\textbf{y}) +\frac{1}{2} \epsilon^2 [ \hat{\phi}(\textbf{x}), \hat{\phi}(\textbf{y}) ] } \ket{0}_{field} \right] \\
\end{align}

\noindent Where the last line is gotten via the Baker-Campbell-Hausdorff relation, since the two fields operators do not necessarily commute, and the interference between the two measurement events is determined by whether the commutator is zero or nonzero, which is calculated \\
\begin{align}
[ \hat{\phi}(\textbf{x}), \hat{\phi}(\textbf{y}) ] &= \int \frac{d^3 p}{(2\pi)^3} \int \frac{d^3 q}{(2\pi)^3} \frac{1}{\sqrt{4 \omega_p \omega_q}} \left( [\hat{a}_p, \hat{a}_q^\dagger ] e^{-i \textbf{p} \cdot \textbf{x} + i \textbf{q} \cdot \textbf{y}} + [\hat{a}_p^\dagger, \hat{a}_q ] e^{i \textbf{p} \cdot \textbf{x} - i \textbf{q} \cdot \textbf{y}} \right) \\
&= \int \frac{d^3 p}{(2\pi)^3} \frac{1}{\sqrt{2 \omega_p}} \left( e^{-i \textbf{p} \cdot (\textbf{x}-\textbf{y})} - e^{i \textbf{p} \cdot (\textbf{x}-\textbf{y})} \right) \cdot \mathbb{I} \\
[ \hat{\phi}(\textbf{x}), \hat{\phi}(\textbf{y}) ] &= \Delta (\textbf{x} - \textbf{y}) \cdot \mathbb{I}
\end{align}

\noindent Where the commutation relation $[\hat{a}_p, \hat{a}_q^\dagger] = (2\pi)^3 \delta^{(3)} (p-q) \cdot \mathbb{I}$ is used, and the quantity $\Delta (\textbf{x} - \textbf{y})$, the \textbf{correlation function}, which is Lorentz invariant, must be zero when $\textbf{x}$ and $\textbf{y}$ are space-like separated, such that $(\textbf{x} - \textbf{y})^2 < 0$. \\

\noindent Consider the space-like separation, $(\textbf{x} - \textbf{y})^2 < 0$, and enter a reference frame where $\textbf{x} - \textbf{y} = (0, x-y)$, and the correlation function $\Delta (\textbf{x} - \textbf{y})$ becomes zero!
\begin{align}
\Delta (\textbf{x} - \textbf{y}) &= \frac{1}{2} \int \frac{d^3 p}{(2\pi)^3} \frac{1}{\sqrt{|p|^2+m^2}} \left( e^{i p \cdot (x-y)} - e^{-ip \cdot (x-y)} \right) \\
&= \frac{1}{2} \int \frac{d^3 p}{(2\pi)^3} \frac{1}{\sqrt{|p|^2+m^2}} \left( e^{i p \cdot (x-y)} - e^{ip \cdot (x-y)} \right) \\
\Delta (\textbf{x} - \textbf{y}) &= 0
\end{align}

\noindent Where the second line is gotten by applying the Lorentz transformation $(x-y) \rightarrow -(x-y)$, which is allowed for in a space-like interval. \\

\begin{figure}[H]
	\centering
	\includegraphics[scale=0.3]{images/inversion.png}
	\caption{Sketch of spatial inversion of spacetime location $\textbf{y}$.}
\end{figure}

\noindent Therefore, information does not travel faster than the speed of light, and the field operators respect causality when they are space-like separated! Note that this does not yet prove that the entire theory is causal. \\

\noindent As a side note, consider the case where the two field measurements are time-like separated, such that $(\textbf{x} - \textbf{y})^2 > 0$, and enter a reference frame where $\textbf{x} - \textbf{y} = (t, 0, 0, 0)$. In this reference frame of time-like separation, the correlation function can \textbf{not} be zero
\begin{align}
\Delta (\textbf{x} - \textbf{y}) &= \int \frac{d^3 p}{(2\pi)^3} \frac{1}{\sqrt{2 \omega_p}} \left( e^{-i\omega_p t} - e^{i\omega_p t} \right) \\
&= \frac{1}{4\pi^2} \int_m^{\infty} dE \, \sqrt{E^2 + m^2} e^{-iEt} \\
\Delta (\textbf{x} - \textbf{y}) &\sim e^{-imt} - e^{imt} \ne 0 .
\end{align}

\noindent Return to consider the (non-physical) two-point correlation function in a space-like interval
\begin{align}
D(\textbf{x}-\textbf{y}) &= \bra{0} \hat{\phi}(\textbf{x}) \hat{\phi}(\textbf{y}) \ket{0} \\
&= \int \frac{d^3 p}{(2\pi)^3} \frac{1}{2 \omega_p} e^{-i \textbf{p} \cdot (\textbf{x}-\textbf{y})} \\
&= \frac{1}{2} \int \frac{d^3 p}{(2\pi)^3} \frac{1}{\sqrt{|p|^2 + m^2}} e^{i p \cdot( x-y)} \\
&= \frac{m}{4\pi^2} \frac{1}{|x-y|} K_1(m|x-y|) \\
D(\textbf{x}-\textbf{y}) &\sim e^{-m|x-y|} \ne 0
\end{align}

\noindent Where $K_1(x)$ denotes the Hankel function. \\

\noindent So, when $\textbf{x}$ and $\textbf{y}$ are space-like separated, this correlation function can not be zero, and can therefore not carry any information, as it would have to travel faster than the speed of light. \\

\noindent An important application of the correlation function is the \textbf{Feynman propagator}, which is used later in perturbative expansions for describing interactions.
\begin{align}
    \Delta_F (\textbf{x} - \textbf{y}) &=
    \begin{cases}
      D(\textbf{x} - \textbf{y}), & x^0 > y^0 \\
      D(\textbf{y} - \textbf{x}), & x^0 < y^0
    \end{cases} \\
\Delta_F (\textbf{x} - \textbf{y}) &= \bra{0} \mathcal{T} [ \hat{\phi}(\textbf{x}) \hat{\phi}(\textbf{y}) ] \ket{0}
\end{align}

\noindent Where $\mathcal{T[\,]}$ is called the \textbf{time-ordering operator}
\begin{equation}
\mathcal{T} [ \hat{\phi}(\textbf{x}) \hat{\phi}(\textbf{y}) ] =
    \begin{cases}
      \hat{\phi}(\textbf{x}) \hat{\phi}(\textbf{y}), & x^0 > y^0 \\
      \hat{\phi}(\textbf{y}) \hat{\phi}(\textbf{x}), & x^0 < y^0
    \end{cases}
\end{equation}

\noindent Another important definition of the Feynman propagator is in terms of complex variables and contour integrals. \\

\noindent \textbf{Lemma:} 
\begin{align}
\Delta_F (\textbf{x} - \textbf{y}) &= \int \frac{d^4 p}{(2\pi)^4} \frac{i e^{-i \textbf{p} \cdot (\textbf{x}-\textbf{y})}}{|\textbf{p}|^2 - m^2 +i \epsilon} \\
&= \int \frac{d^3 p}{(2\pi)^3} \int_C \frac{d p^0}{2\pi} \frac{i e^{-i \textbf{p} \cdot (\textbf{x}-\textbf{y})}}{|\textbf{p}|^2 - m^2 +i \epsilon}
\end{align}

\noindent Where the integrand of the contour integral over the complex variable $p^0$ has two poles at $\pm i\epsilon$, and the contour is taken along the real $p^0$ axis, and closed in the upper or lower half-plane, depending on the value of $p^0$. \\

\begin{figure}[H]
	\centering
	\includegraphics[scale=0.5]{images/poles.png}
	\caption{Poles of the Feynman propagator, where the contour may be taken in the upper half-plane for $t<0$, and in the lower half-plane for $t>0$, where $t=x^0-y_0$.}
\end{figure}

\noindent Lastly, an observation that the Feynman propagator, which is related thus far to the two-point correlation function, is also a Green's function (inverse of a differential operator) for the Klein-Gordon partial differential equation
\begin{align}
(\partial_0^2 - \nabla^2 + m^2) \Delta_F(\textbf{x} - \textbf{y}) &= -i \delta^{(4)} (\textbf{x} - \textbf{y}) \\
\sim \hat{\mathbb{L}} \cdot \hat{\mathbb{L}}^{-1} &= \mathbb{I}
\end{align}

\noindent Where the left-hand side is the product of a linear differential operator, the Klein-Gordon operator, and its inverse, the Feynman propagator, and the right-hand side of the equation is, in essence, the identity.

\clearpage

\section{Lecture 7: Representing Symmetries in QFT}
\label{sec:lec7}

\noindent Here we finish study of the quantum Klein-Gordon field by working out how the Lorentz group is unitarily represented on the space of states on the Klein-Gordon field. Recall that for a relativistic quantum field theory, we must have a (projective) unitary representation of the Poincar\'e group $U(\Lambda, a)$ on a Hilbert space $\mathcal{H}$. Thus far, we have constructed the Hilbert space, with respect to some norm $|| \cdot ||$, as the space of states gotten by applying creation operators to the vacuum state

\begin{equation}
\mathcal{H} = span \{ \hat{a}_{\textbf{p}_1}^\dagger \hat{a}_{\textbf{p}_2}^\dagger \dots \hat{a}_{\textbf{p}_n}^\dagger \ket{\Omega} \} ^{||\cdot||}
\end{equation}

\subsection*{Digression: Continuous Groups of Symmetries}

\noindent The central idea of a group with a continuous manifold structure (e.g., Lie groups) is to study symmetries close to, localized to, the identity and then exponentiate to larger, more global, elements. Important continuous operations on this manifold $\mathcal{M}$ include (closed) composition and inverse. \\

\begin{figure}[H]
	\centering
	\includegraphics[scale=0.3]{images/groupmanifold.png}
	\caption{Sketch of a group manifold with identity $\mathbb{I}$ and an element of the manifold $g$.}
\end{figure}

\begin{align}
\text{Closure:} \,\,\,\,\, \mathcal{M} \times \mathcal{M} &\rightarrow \mathcal{M} \\
g \times h &\rightarrow g \cdot h \\
\text{Inverse:} \,\,\,\,\,\,\,\,\,\,\,\,\,\,\,\,\,\,\,\, \mathcal{M} &\rightarrow \mathcal{M} \\
g &\rightarrow g^{-1}
\end{align}

\noindent The Poincar\'e group is an example of a continuous Lie group, and to understand its structure, consider elements $g \in \mathcal{M}$ infinitesimally close to the identity $\mathbb{I}$, such that $g - \mathbb{I} \sim \mathcal{O}(\epsilon)$. These infinitesimal elements are elements of the tangent space $T_{\mathbb{I}} \, \mathcal{M}$, which is a linear space and is called a \textit{Lie algebra}. The basis vectors of $T_{\mathbb{I}} \, \mathcal{M}$ are written as $x^j$, $j=1,2,\dots,dim(\mathcal{M})$.

\subsection*{Examples of Tangent Spaces}

\begin{enumerate}
\item $1 \times 1$ unitary matrices $U(1) = \{ \phi \in \mathbb{C} : |\phi|^2 = 1 \}$
\subitem $\rightarrow T_1 \, U(1) = \{ z: (1+\epsilon z)^* (1+\epsilon z) = 1$ to order $\epsilon$,  s.t. $Re[z]=0) \}$

\item $3 \times 3$ Euclidean rotation matrices $O(3) = \{ O \in \mathcal{M}_3(\mathbb{R}) : O^T O = \mathbb{I} \}$
\subitem $\rightarrow T_{\mathbb{I}} \, O(3) = \{ X : (\mathbb{I} + \epsilon X)^T (\mathbb{I} + \epsilon X) = \mathbb{I}$ to order $\epsilon$, s.t. $X + X^T = 0 \}$
\subitem - Note that including inversions promotes $O(3)$ to $SO(3)$.
\subitem - Basis of $O(3) = \\ \{ X = \sum_{j=1}^3 X_j J^j : J^1 = \left(\begin{smallmatrix} 0&1&0 \\ -1&0&0 \\ 0&0&0 \end{smallmatrix}\right), J^2 = \left(\begin{smallmatrix} 0&0&1 \\ 0&0&0 \\ -1&0&0 \end{smallmatrix}\right), J^3 = \left(\begin{smallmatrix} 0&0&0 \\ 0&0&1 \\ 0&-1&0 \end{smallmatrix}\right) \}$

\item $4 \times 4$ Lorentz transformation matrices $G = \{ \Lambda : \eta_{\mu\nu} \Lambda^\mu_{\,\,\,\rho} \Lambda^\nu_{\,\,\,\sigma} = \eta_{\rho\sigma} \}$ 
\subitem \begin{align*}
\rightarrow T_{\mathbb{I}} \, G &= \{ \omega : \eta_{\mu\nu} (\mathbb{I} + \epsilon \omega)^\mu_{\,\,\,\rho} (\mathbb{I} + \epsilon \omega)^\nu_{\,\,\,\sigma} = \eta_{\rho\sigma} \} \\
&= \{ \omega : \text{antisymmetric, s.t. } \omega^{\mu\nu} = -\omega^{\nu\mu} \} \\
&\,\,\,\,\,\,\, {\scriptsize \text{Note the "upstairs" covariant indices on }} \omega \\
&\,\,\,\,\,\,\, {\scriptsize \text{Using "downstairs" contravariant indices}} \\ 
&\,\,\,\,\,\,\, {\scriptsize \text{requires multiplication by the metric } \eta_{\mu\nu} \text{ as above.}} \\
&= \{ \omega : \text{change of basis to } \omega = \sum_{j=1}^6 \frac{1}{2} \Omega_j J^j \} \\
&\,\,\,\,\,\,\, {\scriptsize \text{Change indices via the bijection } j \rightarrow (\rho \, \sigma), \text{with } 0<\rho<\sigma\le 3 } \\
&\,\,\,\,\,\,\,\,\,\,\,\,\,\,\,\,\,\, 1 \rightarrow (0 \, 1); \, 2 \rightarrow (0 \, 2); \, 3 \rightarrow (0 \, 3) \\
&\,\,\,\,\,\,\,\,\,\,\,\,\,\,\,\,\,\, 4 \rightarrow (1 \, 2); \, 5 \rightarrow (1 \, 3); \, 6 \rightarrow (2 \, 3) \\
&= \tag{\textbf{Exercise}} \{ \omega : \text{with index bijection } \omega = \sum_{(\rho\sigma)} \frac{1}{2} \Omega_{(\rho\sigma)}  J^{(\rho\sigma)}  \} \\
&\,\,\,\,\,\,\,\,\,\,\, {\scriptsize \text{This bijection allows the basis to be expressed }}  \\
&\,\,\,\,\,\,\,\,\,\,\, {\scriptsize \text{by the formula } [J^{(\rho\sigma)}]^\mu_{\,\,\nu} = \eta^{\rho\sigma} \delta^{\sigma}_{\,\,\nu} - \eta^{\sigma\mu} \delta^{\rho}_{\,\,\nu}} \\
\end{align*}
\end{enumerate}

\noindent Understanding the structure of a linear space infinitesimally close to the identity yields information about the whole Lie group and the global structure of the manifold. Also note that multiplication on this linear space is a continuous map that strongly determines the group structure on the manifold. \\

\noindent To demonstrate how a Lie algebra on the (linear) tangent space produces a Lie group, and vice versa, consider the element of the tangent space of a manifold $X \in T_{\mathbb{I}} \, \mathcal{M}$, such that $\mathbb{I} + s X \sim e^{\epsilon X} \in \mathcal{M}$, to order $\epsilon$. \\

\noindent Define the function $g(s) = \lim_{n\to\infty} (\mathbb{I} + \frac{s}{n} X)^n = e^{s X} \in \mathcal{M}$. Therefore, every element of the Lie algebra (tangent space to the identity) determines an element of the Lie group (manifold). To go the other way, and determine the Lie algebra from the Lie group, apply the logarithm map.\\

\begin{figure}[H]
	\centering
	\includegraphics[scale=0.4]{images/exponentiation.png}
	\caption{Figure of exponentiation map on Lie group (e.g., $O(3)$) manifold. Translation from identity a distance $s$ along the $X$ direction. Resulting position on manifold is called the exponential of $s X$.}
\end{figure}

\subsection*{Algebraic structure on tangent space}

\noindent The algebraic structure on the tangent space is defined by the \textit{group commutator} on the manifold, which is also a group. 
\begin{align}
[\, , \,] : \mathcal{M} \times \mathcal{M} &\rightarrow \mathcal{M} \\
(g, h) &\rightarrow [g, h] = ghg^{-1}h^{-1}, \, \forall \, g, h \in \mathcal{M}
\end{align}

\noindent The group commutator also pushes forward to a mapping of the tangent space
\begin{equation}
[\, , \,] : T_{\mathbb{I}} \, \mathcal{M} \times T_{\mathbb{I}} \, \mathcal{M} \rightarrow T_{\mathbb{I}} \, \mathcal{M}
\end{equation}

\noindent Consider the following group commutator which is an element of the manifold
\begin{align}
[\mathbb{I} + \epsilon X, \mathbb{I} + \delta Y] &= (\mathbb{I} + \epsilon X) (\mathbb{I} + \delta Y) (\mathbb{I} - \epsilon X) (\mathbb{I} - \delta Y) \\
&= \mathbb{I} + \epsilon \delta (XY - YX) + \mathcal{O}(\epsilon^2) + \mathcal{O}(\delta^2) + \dots \\
&\sim \mathbb{I} + \epsilon \delta [X, Y]
\end{align}

\noindent Where $\epsilon$ and $\delta$ are small, independent parameters, and the Lie algebra commutator $[X, Y]$ is, therefore, an element of the tangent space of the manifold, such that $[X,Y] \in T_{\mathbb{I}} \, \mathcal{M}$. \\

\subsection*{Examples of Lie algebra commutators}

\begin{enumerate}
\item $U(1)$: trivial
\item $O(3)$: $[J^j, J^k] = -\epsilon^{jk}_{\,\,\, l} J^l$ 
\item Lorentz group: $[J^{\rho\sigma}, J^{\tau\nu}] = \eta^{\sigma\tau}J^{\rho\nu} - \eta^{\rho\tau}J^{\sigma\nu} + \eta^{\rho\nu}J^{\sigma\tau} - \eta^{\sigma\nu}J^{\rho\tau}$

\subitem E.g., \[ [J^{01}]^\mu_{\,\,\, \nu} = 
\left( 
\begin{array}{@{}c|c@{}}
\begin{matrix} 0 & 1 \\ 1 & 0 \end{matrix} & \bigzero \\
\hline
\bigzero & \begin{matrix} 0 & 0 \\ 0 & 0 \end{matrix} 
\end{array} \right), \,\, [J^{12}]^\mu_{\,\,\, \nu} = 
\left( 
\begin{array}{@{}c|c@{}}
\bigzero & \begin{matrix} 0 & 0 \\ -1 & 0 \end{matrix} \\
\hline
\begin{matrix} 0 & 1 \\ 0 & 0 \end{matrix} & \bigzero
\end{array} \right) \]

\subitem \[ [J^{01}, J^{12}] = -J^{13} =  
\left( 
\begin{array}{@{}c|c@{}}
\bigzero & \begin{matrix} 0 & 0 \\ 0 & -1 \end{matrix} \\
\hline
\begin{matrix} 0 & 0 \\ 0 & 1 \end{matrix} & \bigzero
\end{array} \right) \]

\subitem Special names for this particular Lie group elements in the Lorentz transformation
\subsubitem \textbf{Generators of boosts} (pure boost by exponentiating, s.t. $e^{\frac{1}{2} \epsilon_j K^j}$): 
\subsubitem $J^{0j} = K^j$
\subsubitem \textbf{Generators of rotations} (elements of $O(3) \subset$ Lorentz group): 
\subsubitem $J^{12} = J^1, \, J^{13} = J^2, \, J^{23} = J^3$
\end{enumerate}

\subsection*{Representations of Lie groups}

Consider a representation $\pi$ of the Lie algebra $T_\mathbb{I}\,\mathcal{M}$ which is a linear map from the tangent space to the Hilbert space of bounded linear operators, such that the Lie bracket property is preserved, such that $[\pi(X), \pi(Y)] = \pi([X,Y])$
\begin{equation}
\pi : T_\mathbb{I}\,\mathcal{M} \rightarrow \mathcal{B} (\mathcal{H})
\end{equation}

\noindent This representation is exponentiated to a representation of the Lie group manifold $\mathcal{M}$
\begin{equation}
\pi(g = e^{\epsilon X}) = e^{\epsilon \pi(X)}
\end{equation}

\noindent This shows that one can either try to find matrices that obey the Lie group law, or, more easily, focus on the Lie algebra (linear space) and find matrices that obey the Lie bracket property; Lie group representations are often not worked with directly, but the elements of the Lie algebra can just be exponentiated to obtain the representation of the Lie group. \\

\noindent To implement a general Lorentz transformation on the Hilbert space of states allowed in the Klein-Gordon field, Noether's theorem, as well as the \textit{inverse Noether's theorem}, is employed. Noether's theorem allows conserved currents to be derived from symmetry transformations. Information in thrown out when integrating the currents over space $\int d^3 x$ to get the conserved charge, but the time-like component is left alone, and information about the structure of the symmetry transformation is conserved, which can be gotten back by the inverse Noether's theorem. \\

\noindent \textbf{Recap of Noether's theorem:} For each symmetry of a field, with respect to a coordinate transformation $x \rightarrow \exp^{\epsilon J ^{\rho\sigma}}$, there exists a conserved current per $J^{\rho\sigma}$. For example, the group of Lorentz transformations is described by six independent parameters, the generators of the transformation, associated with six conserved currents. The conserved currents from the symmetries of the Lorentz transformation have the form
\begin{equation}
\mathcal{J}^{(\rho\sigma)} = \textbf{x}^\rho T^{\mu\sigma} - \textbf{x}^\sigma T^{\mu\rho}
\end{equation}

\noindent Where $\textbf{x}^\mu$ are 4-vector spacetime coordinates, and $T^{\mu\nu}$ are elements of the energy-momentum tensor. The conserved charges are then gotten by integrating over space
\begin{equation}
Q^{(\rho\sigma)} = \int d^3 x \, \mathcal{J}^{(\rho\sigma)} = \int d^3x \,  (\textbf{x}^\rho T^{0\sigma} - \textbf{x}^\sigma T^{0\rho})
\end{equation}
\\
\textit{"Noether's theorem is really just a fancy telescoping series in disguise."} \\

\noindent The arguably more profound statement regarding conserved charges and symmetries is the inverse Noether's theorem. \\

\noindent \textbf{Inverse Noether's theorem:} Conserved charges are the generators, represent the Lie algebra, of the symmetry transformations from which they came, and generate canonical transformations, or representations of the symmetries, on phase space. Classically, 
\begin{equation}
\{Q^{(\rho\sigma)}, Q^{(\tau\nu)}\}_{PB} = \eta^{\sigma\tau}Q^{\rho\nu} - \eta^{\rho\tau}Q^{\sigma\nu} + \eta^{\rho\nu}Q^{\sigma\tau} - \eta^{\sigma\nu}Q^{\rho\tau}
\end{equation}

\noindent Propose that we "just put hats on" the conserved charges, and check that they obey the Lie algebra of the Lorentz group. It turns out that this works for free theories, and we have, at least, one representation of the Lie algebra of the Lorentz group in the context of one, the Klein-Gordon, quantum field.
\begin{equation}
\hat{Q}^{\mu\nu} = \int d^3 x (x^\mu \hat{T}^{0\nu} - x^\nu \hat{T}^{0\mu})
\end{equation}

\noindent Consider the $0j^{th}$ conserved charge, using the specialized notation, and check that $\hat{Q}^{0j} = \hat{K}^j$ does indeed generate boosts and is time independent.
\begin{align}
\hat{Q}^{0j} = \hat{K}^j &= \int d^3 x \, (x^j \hat{T}^{00} - x^0 \hat{T}^{0j}) \\
\hat{K}^j &= -t \, \hat{P}^j + \int d^3 x \, x^j \hat{\mathscr{H}}(t,x) \\
\frac{d \hat{K}^j }{dt} &= -\hat{P}^j + i [\hat{H}, \int d^3 x \, x^j \hat{\mathscr{H}}(t,x) ] \\
0 &= \hat{P}^j + i[\hat{H}, \hat{K}^j] \\
[\hat{H}, \hat{K}^j] &= -i \hat{P}^j .
\end{align}

\noindent Where $\hat{P}^j$ is the total field momentum, and $\hat{\mathscr{H}}$ is the Hamiltonian density, which does not commute with the Hamiltonian $\hat{H}$. The third line cancels the time dependence of the two right-hand side terms. This shows that an infinitesimal shift in time and an infinitesimal boost is equal to an infinitesimal shift in space. \\

\noindent Similarly for the generators of rotation, which are manifestly time independent, it must be checked that they obey the correct Lie algebra. 
\begin{equation}
\hat{J}^{jk} = \int d^3 x \, \hat{\pi}(x)(x^j \partial_k - x^k \partial_j) \hat{\phi}(x)
\end{equation}

\noindent To perform a Lorentz transformation on the Hilbert state space, a unitary operator is created by putting some parameters $\Omega$ in front of the generators of rotation and exponentiating
\begin{equation}
\hat{U}(\Lambda) = e^{-\frac{1}{2} \Omega_{\rho\sigma} \hat{J}^{\rho\sigma}}
\end{equation}

\noindent There are 9 commutation relations, that must be checked (\textbf{Exercises}), that yield the full Lie algebra of the Poincar\'e group.
\begin{align}
[\hat{J}^j, \hat{J}^k] &= -i \epsilon^{jk}_{\,\,\,\,l} \hat{J}^l \\
[\hat{J}^j, \hat{K}^k] &= -i \epsilon^{jk}_{\,\,\,\,l} \hat{K}^l \\
[\hat{K}^j, \hat{K}^k] &= i \epsilon^{jk}_{\,\,\,\,l} \hat{J}^l \\
[\hat{J}^j, \hat{P}^k] &= -i \epsilon^{jk}_{\,\,\,\,l} \hat{P}^l \\
[\hat{K}^j, \hat{P}^k] &= i \delta^{jk} \hat{H} \\
[\hat{K}^j, \hat{H}] &= i \hat{P}^j \\
[\hat{J}^j, \hat{H}] &=  [\hat{P}^j, \hat{H}] = [\hat{P}^j, \hat{P}^k] = 0
\end{align}

\noindent This now demonstrates how the Klein-Gordon field gives a full (Lie algebra) representation of the Poincar\'e group, and  proves that to perform a Poincar\'e transformation on a state of Klein-Gordon particles, one simply applies a unitary transformation via exponentiation of the above operators, which are the generators of transformations.

\clearpage

\section{Lecture 8: Interactions in QFT}
\label{sec:lec8}

\noindent Thus far, we have studied the Klein-Gordon quantum field, which evolves with time in the Heisenberg picture via the Klein Hamiltonian $H_{KG}$, the generator of time translations
\begin{equation}
\hat{\phi}(t,x) = e^{i\hat{H}_{KG}t} \hat{\phi}(0,x) e^{-i\hat{H}_{KG}t}.
\end{equation}

\noindent We have obtained a full unitary representation of the Poincar\'e group for the Klein-Gordon field by constucting a space of states in terms of the field position operator $\hat{\phi}$ and the field momentum operator $\hat{\pi}$ via the generator of time translation $\hat{H}$, the generators of spatial translation $\hat{P}^j$, and the conserved charges $\hat{Q}^{\mu\nu}$. This is a \textit{free theory}, where the dynamics of two or more spacetime events evolve completely independently of each other with no interactions between particles and field, and is relatively easy to solve. \\

\noindent To attempt to account for interactions, construct a Hilbert space spanned by states of the form $\{  \hat{a}_p^\dagger \hat{a}_q^\dagger \ket{0} \}$, and add a (spatially localized) momentum distribution
\begin{equation}
\ket{\Phi_2} = \int \frac{d^3 p d^3 q}{(2\pi)^6} \, \phi_x(p) \phi_y(q) \cdot \hat{a}_p^\dagger \hat{a}_q^\dagger \ket{0}
\end{equation}

\noindent This states evolves according to the Hamiltonian
\begin{align}
\ket{\Phi_2(t)} &= e^{-i\hat{H}_{KG}t} \ket{\Phi_2} \\
&= \int \frac{d^3 p d^3 q}{(2\pi)^6} \, \phi_x(p) \phi_y(q) e^{-i\hat{H}_{KG}t} \hat{a}_p^\dagger e^{i\hat{H}_{KG}t} e^{-i\hat{H}_{KG}t}  \hat{a}_q^\dagger e^{i\hat{H}_{KG}t}  \ket{0}
\end{align}

\noindent Where $\hat{H}_{KG}$ is quadratic in the creation operators $\hat{a}_p^\dagger$, meaning that the quantity $e^{-i\hat{H}_{KG}t} \hat{a}_p^\dagger e^{i\hat{H}_{KG}t}$ is linear in the creation operators $\hat{a}_p^\dagger$. Therefore, the particles eveolve independently of each other in this attempted formalism, and there are no interactions, which is unphysical for an interacting theory. \\

\noindent Desired characteristics of the interactions that we are attempting to describe are
\begin{enumerate}
\item Model physical experiments
\item Maintain Lorentz invariance
\item Local interactions
\end{enumerate}

\noindent To fulfill these characteristics, we consider studying models with (classical) Lagrangian densities of the form
\begin{equation}
\mathcal{L} = \frac{1}{2} (\partial_\mu \phi(x))(\partial^\mu \phi(x)) - \frac{1}{2}m^2\phi^2(x) - \sum_{n\ge 3}^\infty \frac{\lambda_n}{n!} \phi^n(x)
\end{equation}

\noindent It will later be shown that Lagrangian densities with $n>4$ are irrelevant to observable physics, and $n=3$ leads to instabilities, and neither case is renormalizable. Therefore, the only relevant \textit{interacting scalar quantum field theory} is the $n=4$ case
\begin{equation}
\mathcal{L} = \frac{1}{2} (\partial_\mu \phi(x))(\partial^\mu \phi(x)) - \frac{1}{2}m^2\phi^2(x) - \frac{\lambda}{4!} \phi^4(x)
\end{equation}

\noindent In a quantum field theory, interactions are handled in several ways
\begin{enumerate}
\item Perturbation theory
	\subitem {\small Expand Hamiltonian in Taylor series in terms of a small parameter} 
	\subitem {\small Leads to a solvable model when this parameter is set to zero}
	\subitem {\small Feynman diagrams systematically handle all interactions in infinite series}
\item Variational methods
	\subitem {\small Approximates the system and minimizes error parameters}
\item Monte Carlo sampling
\item Exact solutions
	\subitem {\small Bethe Ansatz in $(1+1)$ dimensions}
	\subitem {\small Topological QFT in $(2+1)$ dimensions}
	\subitem {\small Supersymmetry in higher dimensions}
	\subitem {\small Large N limit}
\end{enumerate}

\subsection*{Perturbation theory}

\noindent Consider the "small" addition $\hat{H}_{int}$ to the free theory Hamiltonian $\hat{H}_0$ to make the full Hamiltonian $\hat{H}$
\begin{equation}
\hat{H} = \hat{H}_0 + \hat{H}_{int}
\end{equation}

\noindent Technically, we demand that $||\hat{H}_{int}||_\infty << 1$, but it often happens that $||\hat{H}_{int}||_\infty \rightarrow \infty$, where $||\hat{H}_{int}||_\infty$ is the largest eigenvalue that dominates the error estimates. Therefore, we pretend that $||\hat{H}_{int}||_\infty << 1$, and solve the time-dependent Schr\"odinger equation 
\begin{equation}
i\frac{d}{dt}\ket{\psi} = \hat{H}\ket{\psi}
\end{equation}

\subsection*{Interaction picture}

\noindent Enter a new reference frame, the \textbf{interaction picture}, or the \textbf{Heisenberg picture}, where all states and operators from the Schr\"odinger picture, denoted by subscript "$S$", are transformed via
\begin{align}
\ket{\psi_I (t)} &= e^{i\hat{H}_0 t} \ket{\psi_S (t)} \\
\mathcal{O} &= e^{i\hat{H}_0 t} \, \mathcal{O}_S  \, e^{-i\hat{H}_0 t}
\end{align}

\noindent The time evolution on the interaction space of states is then
\begin{align}
i \frac{d}{dt} \ket{\psi_I (t)} &=  i \frac{d}{dt} e^{i\hat{H}_0 t} \ket{\psi_S (t)} \\
&= - \hat{H}_0 e^{i\hat{H}_0 t} \ket{\psi_S (t)}  + i e^{i\hat{H}_0 t} \frac{d}{dt} \ket{\psi_S (t)} \\
&= - \hat{H}_0 e^{i\hat{H}_0 t} \ket{\psi_S (t)} + i e^{i\hat{H}_0 t}(-i\hat{H}\ket{\psi_S (t)}) \\
&= (- \hat{H}_0 e^{i\hat{H}_0 t} + e^{i\hat{H}_0 t} (\hat{H}_0 + \hat{H}_{int})) \ket{\psi_S (t)} \\
&= (\cancel{- \hat{H}_0 e^{i\hat{H}_0 t}} + e^{i\hat{H}_0 t} (\cancel{\hat{H}_0} + \hat{H}_{int})) \cdot e^{-i\hat{H}_0 t} e^{i\hat{H}_0 t} \cdot \ket{\psi_S (t)} \\
i \frac{d}{dt} \ket{\psi_I (t)} &= (\hat{H}_{int})_I (t) \ket{\psi_I (t)} 
\end{align}

\noindent Note: From here we drop the subscript "$I$" on the interaction Hamiltonian 
\begin{equation}
(\hat{H}_{int})_I (t) \rightarrow \hat{H}_{int} (t).
\end{equation}

\noindent The interacting time-dependent solution is
\begin{equation}
\ket{\psi_I (t)}  = \hat{U}(t,t_0) \ket{\psi_I (t_0)}. 
\end{equation}

\noindent Where the operator $\hat{U}(t,t_0)$ is the propagator, and satisfies the equation
\begin{equation}
i \frac{d}{dt} \hat{U}(t,t_0) = \hat{H}_{int} (t) \, \hat{U}(t,t_0)
\end{equation}

\noindent Integrating this equation with respect to $t$ yields a constraint on the propagator
\begin{equation}
\hat{U}(t,t_0) = \mathbb{I} - i \int_{t_0}^t dt' \, \hat{H}_{int} (t') \hat{U}(t',t_0)
\end{equation}

\noindent One way to solve for $\hat{U}(t,t_0)$ is to guess a solution and check if both sides of the constraint equation are equal. \\

\noindent Another approach is through \textit{fixed point iteration}
\begin{enumerate}
\item Make a guess for $\hat{U}(t,t_0)$
\item Evaluate how wrong it is
\item Minimize error by adding and/or modifying terms to guess
\item Repeat, by substituting the old right-hand side into the new right-hand side, until the left-hand side and the right-hand side of the constraint approaach each other
\end{enumerate}

\noindent The repeated substitution of the propagator into the constraint equation produces the \textbf{Dyson series}, where the $n^{th}$ has the form
\begin{equation}
\hat{U}(t,t_0) = (-i)^n \int^t_{t_0} dt' \int^{t'}_{t_0} dt'' \dots \int^{t_{(n-1)}}_{t_0} dt^{(n-1)} \, \hat{H}_{int}(t') \hat{H}_{int}(t'') \dots \hat{H}_{int}(t^{(n-1)})
\end{equation}

\noindent By the triangle inequality and the product inequality, the norm of the $n^{th}$ term has an upper bound
\begin{equation}
|| \int \dots ||_{\infty} \le \frac{(t-t_0)^n}{n!} (|| \hat{H}_{int} ||_\infty^*)^n .
\end{equation}

\noindent And, since $\hat{H}_{int}(t)$ is just unitarily rotated from $\hat{H}_{int}$
\begin{equation}
|| \hat{H}_{int} ||_\infty^* = \sup_{t' \in [t,t_0]} || \hat{H}_{int}(t') ||_\infty = (||\hat{H}_{int}||_\infty)_S .
\end{equation}

\noindent If $||\hat{H}_{int}||_\infty << 1$, the series has a nonzero radius of convergence. \\

\noindent \textbf{Theorem:} $\hat{U}(t,t_0) = \mathcal{T}[ e^{-i \int^t_{t_0} dt' \, \hat{H}_{int}(t')}]$, where $\mathcal{T}[\,]$ is the time-ordering operator. \\

\noindent To prove, expand the right-hand side in a Taylor series, apply time-ordering, and check that the two sides are equal. \\

\subsection*{Observables in QFT}

\noindent An important observable in QFT is scattering cross sections in scattering experiments. Namely, the $\mathcal{S}$-matrix is determined by Green's functions ($n$-point correlation functions)
\begin{equation}
G^{(n)}(x_1,x_2,\dots,x_n) = \bra{\Omega} \mathcal{T} [ \hat{\phi}_{1H} \hat{\phi}_{2H} \dots \hat{\phi}_{nH} ] \ket{\Omega}
\end{equation}

\noindent Where $\ket{\Omega}$ is the vacuum state of the Hamiltonian, and the subscript "$H$" denotes the Heisenberg picture, such that $\hat{\phi}_{jH} = \hat{\phi}(\textbf{x}_j) = \hat{\phi}(t_j,x_j)$. \\

\noindent \textbf{Claim:} 
\begin{equation}
G^{(n)}(x_1,x_2,\dots,x_n) = \frac{ \bra{0} \mathcal{T}[ \hat{\phi}_{1I} \hat{\phi}_{2I} \dots \hat{\phi}_{nI}\hat{\mathcal{S}}] \ket{0}}{\bra{0} \hat{\mathcal{S}} \ket{0}}
\end{equation}

Where $\bra{\phi}\hat{\mathcal{S}}\ket{\psi} = \lim_{t_\pm \rightarrow \pm \infty} \bra{\phi} \hat{U}(t_+, t_-) \ket{\psi}$, and $\hat{H}_0\ket{0} = 0$. \\

\noindent \textbf{Proof:} \\
\noindent Assume that $t_1>t_2>\dots>t_n$. \\
\noindent Then the right-hand side of the numerator reads
\begin{align}
&\bra{0} \hat{U}(\infty,t_1) \hat{\phi}_{1I} \hat{U}(t_1,t_2) \hat{\phi}_{2I} \dots \hat{\phi}_{nI} \hat{U}(t_n,-\infty) \ket{0} \\
&= \bra{0} \hat{U}(\infty,t_1) \hat{\phi}_{1H} \hat{\phi}_{2H} \dots \hat{\phi}_{nH} \hat{U}(t_0,-\infty) \ket{0}
\end{align}
\noindent Where $\hat{\phi}_H(t,x) = \hat{U}^\dagger(t,t_0) \hat{\phi}_I(t,x) \hat{U}(t,t_0)$ \\
\noindent Now dealing with 
\begin{align}
\hat{U}(t_0,-\infty) \ket{0} &= \lim_{t'\to -\infty} \lim_{t \to t_0} e^{i \hat{H}_0(t-t_0)} e^{-i \hat{H}(t-t')} e^{-i \hat{H}_0(t'-t)} \ket{0} \\
&= \lim_{t' \to -\infty} (\ket{\Omega}\bra{\Omega} + \sum_{n>0} e^{-i E_n (t_0 - t')} \ket{E_n}\bra{E_n})\ket{0}
\end{align}
\noindent Where the nonvanishing terms are written in the eigenbasis of $\hat{H}$. \\
\noindent Quantum fields have a continuous spectra, such that $\sum_{n>0} \sim \int dE$. \\
\noindent Invoke the Riemann-Lebesgue Lemma ($\lim_{k\to\infty} \hat{\psi}(k) = \lim_{k\to\infty} \int \psi(x)e^{ikx} dx = 0$), and consider
\begin{equation}
\lim_{t' \to -\infty} \int dE \braket{E|0} e^{-iE(t-t_0)} \braket{\phi|E} = 0
\end{equation}
\noindent The numerator is now equal to 
\begin{equation}
\braket{0|\Omega}\braket{\Omega|0}\bra{\Omega}\hat{\phi}_{1H}\dots\hat{\phi}_{nH}\ket{\Omega}
\end{equation}
\noindent Where $\braket{0|\Omega}\braket{\Omega|0}$ is equal to the denominator and cancels. QED.\\

\noindent Essentially, interacting quantum field theories come down to throwing in a Taylor series for the $\mathcal{S}$-matrix and $\hat{\phi}_{iI}$, and truncating some terms.

\clearpage

\section{Lecture 9: Interactions and Feynman Diagrams}
\label{sec:lec9}

\noindent The story so far
\begin{itemize}
\item Built a free (non-interacting) relativistic quantum field theory, namely, the Klein-Gordon field, with Hamiltonian $\hat{H}_{KG}$.

\item Added (Lorentz invaraint) interactions via the Hamiltonian
	\subitem \begin{equation} \hat{H}_{int} = \frac{\lambda}{4!} \int d^3 x \, \hat{\phi}^4(\textbf{x}), \, \lambda >> 1 \end{equation}.

\item Used perturbation theory to solve the Hamiltonian $\hat{H} = \hat{H}_{KG} + \hat{H}_{int}$.

\item Studied observable quantities via the $n$-point correlation function
	\subitem \begin{equation} G^{(n)}(\textbf{x}_1, \dots, \textbf{x}_n) = \bra{\Omega} \mathcal{T}[\hat{\phi}_H(\textbf{x}_1) \dots \hat{\phi}_H(\textbf{x}_n) ] \ket{\Omega} \end{equation}.
	\subitem Where $\ket{\Omega}$ is the full, interacting vacuum state, and $\hat{\phi}_H(\textbf{x}_i) = \hat{\phi}_{iH} = \hat{\phi}_I(\textbf{x}_i) = \hat{\phi}_{iI}$ is the field operator in the Heisenberg, interaction, picture, where the observables (e.g., field operators) closer to direct observation.

\item Claimed and "proved" that the $n$-point correlation function can be calculated in terms of the ground state expectation values of the field operators, and the scattering $\mathcal{S}$-matrix .
	\subitem \begin{equation} G^{(n)}(\textbf{x}_1, \dots, \textbf{x}_n) = \frac{\bra{0} \mathcal{T}[\hat{\phi}_{1I} \dots \hat{\phi}_{nI} \mathcal{S} ]\ket{0} }{\bra{0} \mathcal{S} \ket{0}} \end{equation}
	\subitem Where $\bra{\phi}\mathcal{S}\ket{\psi} = \lim_{t\to\infty} \bra{\phi} \mathcal{T}[e^{-i \int^t_{-t} \hat{H}_{int}(t') dt'}] \ket{\psi}, \, \forall \ket{\phi}, \ket{\psi}$
\end{itemize}

\noindent Now, to calculate quantities like the numerator of the $n$-point correlation function consider the field operator in the interaction picture (dropping the subscript "$I$")
\begin{equation}
\hat{\phi}_I(\textbf{x}) = \hat{\phi}(\textbf{x}) = \int \frac{d^3 p}{\sqrt{2 \omega_p}} (\hat{a}_\textbf{p} e^{-i \textbf{p} \cdot \textbf{x}} + \hat{a}^\dagger_\textbf{p} e^{i \textbf{p} \cdot \textbf{x}} ) = \hat{\phi}^+(\textbf{x}) + \hat{\phi}^-(\textbf{x})
\end{equation}

\noindent Where $\textbf{p} = (\omega_p, p)$, with $\omega_p = \sqrt{p^2 + m^2}$, such that $\textbf{p}\cdot\textbf{x} = p^0x^0 - p\cdot x$, and the newly defined operators annihilate the ground state, such that 
\begin{equation} 
\hat{\phi}^+(\textbf{x})\ket{0} = 0 \,\,\,\, \text{and} \,\,\,\, \bra{0}\hat{\phi}^-(\textbf{x}) = 0. 
\end{equation}

\noindent For example, consider the time-ordering of the two particle case (note the notation change for the four-vector in this section $\textbf{x} \rightarrow x$)
\begin{align*}
\mathcal{T}[\hat{\phi}(x)\hat{\phi}(y)]_{x^0>y^0} &= \hat{\phi}(x)\hat{\phi}(y) \\
&= \hat{\phi}^+(x)\hat{\phi}^+(y) + \hat{\phi}^+(x)\hat{\phi}^-(y) \\
&\,\,\,\,\,\, + \hat{\phi}^-(x)\hat{\phi}^+(y) + \hat{\phi}^-(x)\hat{\phi}^-(y) \\
&= \hat{\phi}^+(x)\hat{\phi}^+(y) + \left(\hat{\phi}^-(y)\hat{\phi}^+(x) + [\hat{\phi}^+(x), \hat{\phi}^-(y)]\right) \\
&\,\,\,\,\,\, + \hat{\phi}^-(x)\hat{\phi}^+(y) + \hat{\phi}^-(x)\hat{\phi}^-(y)  \\ 
\mathcal{T}[\hat{\phi}(x)\hat{\phi}(y)]_{x^0>y^0} &= \hat{\phi}^+(x)\hat{\phi}^+(y) + \left(\hat{\phi}^-(y)\hat{\phi}^+(x) + D(x-y)\cdot\mathbb{I}\right) \\
&\,\,\,\,\,\, + \hat{\phi}^-(x)\hat{\phi}^+(y) + \hat{\phi}^-(x)\hat{\phi}^-(y) 
\end{align*}

\noindent Then the ground state expectation value of two interacting field operators is simply the Feynman propagator
\begin{align}
\bra{0} \mathcal{T}[\hat{\phi}(x)\hat{\phi}(y)] \ket{0} &= \Delta_F(x-y) \\
&= i \int \frac{d^4 p}{(2\pi)^4} \frac{e^{-i p\cdot(x-y)}}{p^2 - m^2 + i\epsilon} \,\, , \,\, \epsilon>0 \\
&=
	\begin{cases}
      		D(x-y), & x^0 > y^0 \\
      		D(y-x), & x^0 \le y^0
    	\end{cases}
\end{align}

\subsection*{Wick Contraction and Normal Ordering}

\noindent Introduce some notation for extracting Feynman propagators from quantities like the expectation value of time-ordered field operators, called the \textbf{Wick contraction}. For two field operators, $\hat{\phi}(x)$ and $\hat{\phi}(y)$, and any three other operators $\hat{A}$, $\hat{B}$, and $\hat{C}$, write 
\begin{align}
\wick[offset=1.5em]{\c {\hat{\phi}(x)} \c {\hat{\phi}(y)}} &= \Delta_F(x-y) \cdot \mathbb{I} \\
\wick[offset=1.5em]{A \c {\hat{\phi}(x)} B \c {\hat{\phi}(y)} C} &= \Delta_F(x-y) \cdot \hat{A}\hat{B}\hat{C}
\end{align}

\noindent Also introduce \textbf{normal ordering}, denoted by $\mathcal{N}[\,]$ that sends all "dagger" operators to the left. For example,
\begin{equation}
\mathcal{N}[\hat{a}_p \hat{a}_q^\dagger \hat{a}_r \hat{a}_s^\dagger ] = \hat{a}_q^\dagger \hat{a}_s^\dagger \hat{a}_p \hat{a}_r
\end{equation}

\noindent Observe the relationship between time-ordering and normal-ordering using the Wick contraction
\begin{align}
\mathcal{T}[\hat{\phi}(x) \hat{\phi}(y)] &= \mathcal{N}[\hat{\phi}(x) \hat{\phi}(y) + \Delta_F(x-y)\cdot\mathbb{I}] \\
&= \mathcal{N}[\hat{\phi}(x) \hat{\phi}(y) + \wick[offset=1.5em]{\c {\hat{\phi}(x)} \c {\hat{\phi}(y)}}
\end{align}

\noindent A more involved example of the time-ordering of four field operators, where the only nonzero terms at the end of acting on states will be the "double contractions", since 
\begin{align*}
\mathcal{T}[\hat{\phi}(x_1) \hat{\phi}(x_2) \hat{\phi}(x_3) \hat{\phi}(x_4)] &= \mathcal{N}[\hat{\phi}_1 \hat{\phi}_2 \hat{\phi}_3 \hat{\phi}_4 + \text{"all possible contractions"}]\\
&= \mathcal{N}[\hat{\phi}_1 \hat{\phi}_2 \hat{\phi}_3 \hat{\phi}_4 + \wick[offset=1.5em]{\c {\hat{\phi}_1} \c {\hat{\phi}_2} \hat{\phi}_3 \hat{\phi}_4} + \wick[offset=1.5em]{\c {\hat{\phi}_1} {\hat{\phi}_2} \c {\hat{\phi}_3} \hat{\phi}_4} \\
&\,\,\,\,\,\, + \wick[offset=1.5em]{\c {\hat{\phi}_1} {\hat{\phi}_2} {\hat{\phi}_3} \c {\hat{\phi}_4}} + + \wick[offset=1.5em]{{\hat{\phi}_1} \c {\hat{\phi}_2} \c {\hat{\phi}_3} {\hat{\phi}_4}} + \wick[offset=1.5em]{{\hat{\phi}_1} {\hat{\phi}_2} \c {\hat{\phi}_3} \c {\hat{\phi}_4}} + \wick[offset=1.5em]{{\hat{\phi}_1} \c {\hat{\phi}_2} {\hat{\phi}_3} \c {\hat{\phi}_4}} \\
&\,\,\,\,\,\, + \wick[offset=1.5em]{\c1 {\hat{\phi}_1} \c2 {\hat{\phi}_2} \c1 {\hat{\phi}_3} \c2 {\hat{\phi}_4}} + \wick[offset=1.5em]{\c1 {\hat{\phi}_1} \c2 {\hat{\phi}_2} \c2 {\hat{\phi}_3} \c1 {\hat{\phi}_4}} + \wick[offset=1.5em]{\c1 {\hat{\phi}_1} \c1 {\hat{\phi}_2} \c2 {\hat{\phi}_3} \c2 {\hat{\phi}_4}} ]
\end{align*}

\noindent The ground state matrix elements of $\mathcal{T}[\hat{\phi}_1 \hat{\phi}_2 \hat{\phi}_3 \hat{\phi}_4]$ is then
\begin{align*}
\bra{0} \mathcal{T}[\hat{\phi}_1 \hat{\phi}_2 \hat{\phi}_3 \hat{\phi}_4] \ket{0} &= \cancel{\bra{0} \mathcal{N}[\hat{\phi}_1 \hat{\phi}_2 \hat{\phi}_3 \hat{\phi}_4] \ket{0}} + \cancel{\bra{0} \mathcal{N}[\wick[offset=1.5em]{\c {\hat{\phi}_1} \c {\hat{\phi}_2} \hat{\phi}_3 \hat{\phi}_4}] \ket{0}} + \cancel{\dots} \\
&\,\,\,\,\,\, + \Delta_F(x_1-x_2)\Delta_F(x_3-x_4) + \Delta_F(x_1-x_3)\Delta_F(x_2-x_4) \\
&\,\,\,\,\,\, + \Delta_F(x_1-x_4)\Delta_F(x_2-x_3)
\end{align*}

\noindent Where these are the values of the associated Feynman diagrams, which we write down in a "reverse" way, extracting the diagram from the calculated value. Later, we will extract the values from the diagrams.

\begin{figure}[H]
	\centering
	\includegraphics[scale=0.5]{images/feynman1.png}
	\caption{Feynman diagrams representing the nonzero values in the four particle example above.}
\end{figure}

\noindent Now stated in its general form \\

\noindent \textbf{Wick's Theorem}: $\mathcal{T}[\hat{\phi}_1 \dots \hat{\phi}_n] = \mathcal{N}[\hat{\phi}_1 \dots \hat{\phi}_n + \text{"all possible contractions"}]$. \\

\noindent \textbf{Proof:} \\
\noindent Induct on $n$, with the base case $n=2$ confirmed to be true, and check that the $n-1$ case implies the full $n$ case. \\
\noindent Assume, without loss of generality, that everything is time-ordered, such that $x_1^0 > x_2^0 > \dots > x_n^0$. \\
\noindent Then the left-hand side of Wick's theorem becomes
\begin{equation}
\mathcal{T}[\hat{\phi}_1 \dots \hat{\phi}_n] = \hat{\phi}_1 \hat{\phi}_2 \dots \hat{\phi}_n .
\end{equation}
\noindent Use the inductive hypothesis for $n-1$ case on this equation
\begin{align*}
\hat{\phi}_1 \hat{\phi}_2 \dots \hat{\phi}_n &= (\hat{\phi}_1^+ + \hat{\phi}_1^-) \mathcal{N} [ \hat{\phi}_2 \dots \hat{\phi}_n + \left(\text{\stackanchor{{\small all possible contractions}}{{\small excluding $\hat{\phi}_1$.}}}\right) ] \\
&= \hat{\phi}_1^+ \mathcal{N}[\hat{\phi}_2 \dots \hat{\phi}_n + \left(\text{\stackanchor{{\small all possible contractions}}{{\small excluding $\hat{\phi}_1$.}}}\right)] \\
&\,\,\,\,\,\, + \mathcal{N}[ \hat{\phi}_1^- \hat{\phi}_2 \dots \hat{\phi}_n + \hat{\phi}_1^- \left(\text{\stackanchor{{\small all possible contractions}}{{\small excluding $\hat{\phi}_1$.}}}\right) ] .
\end{align*}
\noindent Since $\hat{\phi}_1^-$ is already normal-ordered. \\
\noindent Focus on the first part of the first term of $\hat{\phi}_1 \hat{\phi}_2 \dots \hat{\phi}_n$ above
\begin{align*}
\hat{\phi}_1^+ \mathcal{N}[\hat{\phi}_2 \dots \hat{\phi}_n] &= \mathcal{N}[\hat{\phi}_2 \dots \hat{\phi}_n] \hat{\phi}_1^+ + [\hat{\phi}_1^+, \mathcal{N}[\hat{\phi}_2 \dots \hat{\phi}_n]] \\
&= \mathcal{N}[\hat{\phi}_1^+ \hat{\phi}_2 \dots \hat{\phi}_n] \\
&\,\,\,\, + \mathcal{N}[ [\hat{\phi}_1^+, \hat{\phi}_2]\hat{\phi}_3 \dots \hat{\phi}_n + \hat{\phi}_2 [\hat{\phi}_1^+,\hat{\phi}_3] \hat{\phi}_4 \dots \hat{\phi}_n + \dots + \hat{\phi}_2\hat{\phi}_3\dots\hat{\phi}_{n-1}[\hat{\phi}_1^+,\hat{\phi}_n]] \\
\hat{\phi}_1^+ \mathcal{N}[\hat{\phi}_2 \dots \hat{\phi}_n] &= \mathcal{N}[ \hat{\phi}_1^+ \hat{\phi}_2 \dots \hat{\phi}_n + \wick[offset=1.5em]{\c {\hat{\phi}_1^+} \c {\hat{\phi}_2} \hat{\phi}_3 \dots \hat{\phi}_n} + \wick[offset=1.5em]{\c {\hat{\phi}_1^+} {\hat{\phi}_2} \c {\hat{\phi}_3} \dots \hat{\phi}_n} + \dots + \wick[offset=1.5em]{\c {\hat{\phi}_1^+} {\hat{\phi}_2} \hat{\phi}_3 \dots \c {\hat{\phi}_n}} ] .
\end{align*}
\noindent Where the second equality follows from $[\hat{\phi}_j^\dagger, \hat{\phi}_k] \propto \mathbb{I}$. \\
\noindent Now, focus on the rest of the first term of $\hat{\phi}_1 \mathcal{N}[\hat{\phi}_2 \dots \hat{\phi}_n]$
\begin{align*}
\hat{\phi}_1^+ \mathcal{N}[ \left(\text{\stackanchor{{\small all possible contractions}}{{\small excluding $\hat{\phi}_1$.}}}\right) ] &= [ \hat{\phi}_1^+, \mathcal{N}[\dots]] + \mathcal{N}[\dots]\hat{\phi}_1^+ \\
&= \mathcal{N}[  \left(\text{\stackanchor{{\small all possible contractions}}{{\small including $\hat{\phi}_1^+$.}}}\right) ] + \mathcal{N}[\hat{\phi}_1^+ \left(\text{\stackanchor{{\small all possible contractions}}{{\small excluding $\hat{\phi}_1$.}}}\right)]
\end{align*}

\clearpage

\section{Lecture 10: Feynman Rules for $\varphi^4$ Theory}
\label{sec:lec10}

\input{chapters/lec10.tex}

\clearpage

\section{Lecture 11: Feynman Rules and Vacuum Bubbles}
\label{sec:lec11}

\input{chapters/lec11.tex}

\clearpage

\section{Lecture 12: The $\mathcal{S}$-matrix in $\varphi^4$ Theory}
\label{sec:lec12}

\input{chapters/lec12.tex}

\clearpage

\section{Lecture 13: Feynman Diagram Expansions in $\varphi^4$ Theory}
\label{sec:lec13}

\input{chapters/lec13.tex} 

\clearpage

\section{Lecture 14: The Dirac Field}
\label{sec:lec14}

\input{chapters/lec14.tex} 

\clearpage

\section{Lecture 15: The Dirac Equation and its Solutions}
\label{sec:lec15}

\input{chapters/lec15.tex} 

\clearpage

\section{Lecture 16: The Quantum Dirac Field}
\label{sec:lec16}

\input{chapters/lec16.tex} 

\clearpage

\section{Lecture 17: The Quantum Dirac Field, Continued}
\label{sec:lec17}

\input{chapters/lec17.tex} 

\clearpage

\section{Lecture 18: Quantum Field Theory for Interacting Fermions and Bosons}
\label{sec:lec18}

\input{chapters/lec18.tex} 

\clearpage

\end{document}

