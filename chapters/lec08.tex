\noindent Thus far, we have studied the Klein-Gordon quantum field, which evolves with time in the Heisenberg picture via the Klein Hamiltonian $H_{KG}$, the generator of time translations
\begin{equation}
\hat{\phi}(t,x) = e^{i\hat{H}_{KG}t} \hat{\phi}(0,x) e^{-i\hat{H}_{KG}t}.
\end{equation}

\noindent We have obtained a full unitary representation of the Poincar\'e group for the Klein-Gordon field by constucting a space of states in terms of the field position operator $\hat{\phi}$ and the field momentum operator $\hat{\pi}$ via the generator of time translation $\hat{H}$, the generators of spatial translation $\hat{P}^j$, and the conserved charges $\hat{Q}^{\mu\nu}$. This is a \textit{free theory}, where the dynamics of two or more spacetime events evolve completely independently of each other with no interactions between particles and field, and is relatively easy to solve. \\

\noindent To attempt to account for interactions, construct a Hilbert space spanned by states of the form $\{  \hat{a}_p^\dagger \hat{a}_q^\dagger \ket{0} \}$, and add a (spatially localized) momentum distribution
\begin{equation}
\ket{\Phi_2} = \int \frac{d^3 p d^3 q}{(2\pi)^6} \, \phi_x(p) \phi_y(q) \cdot \hat{a}_p^\dagger \hat{a}_q^\dagger \ket{0}
\end{equation}

\noindent This states evolves according to the Hamiltonian
\begin{align}
\ket{\Phi_2(t)} &= e^{-i\hat{H}_{KG}t} \ket{\Phi_2} \\
&= \int \frac{d^3 p d^3 q}{(2\pi)^6} \, \phi_x(p) \phi_y(q) e^{-i\hat{H}_{KG}t} \hat{a}_p^\dagger e^{i\hat{H}_{KG}t} e^{-i\hat{H}_{KG}t}  \hat{a}_q^\dagger e^{i\hat{H}_{KG}t}  \ket{0}
\end{align}

\noindent Where $\hat{H}_{KG}$ is quadratic in the creation operators $\hat{a}_p^\dagger$, meaning that the quantity $e^{-i\hat{H}_{KG}t} \hat{a}_p^\dagger e^{i\hat{H}_{KG}t}$ is linear in the creation operators $\hat{a}_p^\dagger$. Therefore, the particles eveolve independently of each other in this attempted formalism, and there are no interactions, which is unphysical for an interacting theory. \\

\noindent Desired characteristics of the interactions that we are attempting to describe are
\begin{enumerate}
\item Model physical experiments
\item Maintain Lorentz invariance
\item Local interactions
\end{enumerate}

\noindent To fulfill these characteristics, we consider studying models with (classical) Lagrangian densities of the form
\begin{equation}
\mathcal{L} = \frac{1}{2} (\partial_\mu \phi(x))(\partial^\mu \phi(x)) - \frac{1}{2}m^2\phi^2(x) - \sum_{n\ge 3}^\infty \frac{\lambda_n}{n!} \phi^n(x)
\end{equation}

\noindent It will later be shown that Lagrangian densities with $n>4$ are irrelevant to observable physics, and $n=3$ leads to instabilities, and neither case is renormalizable. Therefore, the only relevant \textit{interacting scalar quantum field theory} is the $n=4$ case
\begin{equation}
\mathcal{L} = \frac{1}{2} (\partial_\mu \phi(x))(\partial^\mu \phi(x)) - \frac{1}{2}m^2\phi^2(x) - \frac{\lambda}{4!} \phi^4(x)
\end{equation}

\noindent In a quantum field theory, interactions are handled in several ways
\begin{enumerate}
\item Perturbation theory
	\subitem {\small Expand Hamiltonian in Taylor series in terms of a small parameter} 
	\subitem {\small Leads to a solvable model when this parameter is set to zero}
	\subitem {\small Feynman diagrams systematically handle all interactions in infinite series}
\item Variational methods
	\subitem {\small Approximates the system and minimizes error parameters}
\item Monte Carlo sampling
\item Exact solutions
	\subitem {\small Bethe Ansatz in $(1+1)$ dimensions}
	\subitem {\small Topological QFT in $(2+1)$ dimensions}
	\subitem {\small Supersymmetry in higher dimensions}
	\subitem {\small Large N limit}
\end{enumerate}

\subsection*{Perturbation theory}

\noindent Consider the "small" addition $\hat{H}_{int}$ to the free theory Hamiltonian $\hat{H}_0$ to make the full Hamiltonian $\hat{H}$
\begin{equation}
\hat{H} = \hat{H}_0 + \hat{H}_{int}
\end{equation}

\noindent Technically, we demand that $||\hat{H}_{int}||_\infty << 1$, but it often happens that $||\hat{H}_{int}||_\infty \rightarrow \infty$, where $||\hat{H}_{int}||_\infty$ is the largest eigenvalue that dominates the error estimates. Therefore, we pretend that $||\hat{H}_{int}||_\infty << 1$, and solve the time-dependent Schr\"odinger equation 
\begin{equation}
i\frac{d}{dt}\ket{\psi} = \hat{H}\ket{\psi}
\end{equation}

\subsection*{Interaction picture}

\noindent Enter a new reference frame, the \textbf{interaction picture}, or the \textbf{Heisenberg picture}, where all states and operators from the Schr\"odinger picture, denoted by subscript "$S$", are transformed via
\begin{align}
\ket{\psi_I (t)} &= e^{i\hat{H}_0 t} \ket{\psi_S (t)} \\
\mathcal{O} &= e^{i\hat{H}_0 t} \, \mathcal{O}_S  \, e^{-i\hat{H}_0 t}
\end{align}

\noindent The time evolution on the interaction space of states is then
\begin{align}
i \frac{d}{dt} \ket{\psi_I (t)} &=  i \frac{d}{dt} e^{i\hat{H}_0 t} \ket{\psi_S (t)} \\
&= - \hat{H}_0 e^{i\hat{H}_0 t} \ket{\psi_S (t)}  + i e^{i\hat{H}_0 t} \frac{d}{dt} \ket{\psi_S (t)} \\
&= - \hat{H}_0 e^{i\hat{H}_0 t} \ket{\psi_S (t)} + i e^{i\hat{H}_0 t}(-i\hat{H}\ket{\psi_S (t)}) \\
&= (- \hat{H}_0 e^{i\hat{H}_0 t} + e^{i\hat{H}_0 t} (\hat{H}_0 + \hat{H}_{int})) \ket{\psi_S (t)} \\
&= (\cancel{- \hat{H}_0 e^{i\hat{H}_0 t}} + e^{i\hat{H}_0 t} (\cancel{\hat{H}_0} + \hat{H}_{int})) \cdot e^{-i\hat{H}_0 t} e^{i\hat{H}_0 t} \cdot \ket{\psi_S (t)} \\
i \frac{d}{dt} \ket{\psi_I (t)} &= (\hat{H}_{int})_I (t) \ket{\psi_I (t)} 
\end{align}

\noindent Note: From here we drop the subscript "$I$" on the interaction Hamiltonian 
\begin{equation}
(\hat{H}_{int})_I (t) \rightarrow \hat{H}_{int} (t).
\end{equation}

\noindent The interacting time-dependent solution is
\begin{equation}
\ket{\psi_I (t)}  = \hat{U}(t,t_0) \ket{\psi_I (t_0)}. 
\end{equation}

\noindent Where the operator $\hat{U}(t,t_0)$ is the propagator, and satisfies the equation
\begin{equation}
i \frac{d}{dt} \hat{U}(t,t_0) = \hat{H}_{int} (t) \, \hat{U}(t,t_0)
\end{equation}

\noindent Integrating this equation with respect to $t$ yields a constraint on the propagator
\begin{equation}
\hat{U}(t,t_0) = \mathbb{I} - i \int_{t_0}^t dt' \, \hat{H}_{int} (t') \hat{U}(t',t_0)
\end{equation}

\noindent One way to solve for $\hat{U}(t,t_0)$ is to guess a solution and check if both sides of the constraint equation are equal. \\

\noindent Another approach is through \textit{fixed point iteration}
\begin{enumerate}
\item Make a guess for $\hat{U}(t,t_0)$
\item Evaluate how wrong it is
\item Minimize error by adding and/or modifying terms to guess
\item Repeat, by substituting the old right-hand side into the new right-hand side, until the left-hand side and the right-hand side of the constraint approaach each other
\end{enumerate}

\noindent The repeated substitution of the propagator into the constraint equation produces the \textbf{Dyson series}, where the $n^{th}$ has the form
\begin{equation}
\hat{U}(t,t_0) = (-i)^n \int^t_{t_0} dt' \int^{t'}_{t_0} dt'' \dots \int^{t_{(n-1)}}_{t_0} dt^{(n-1)} \, \hat{H}_{int}(t') \hat{H}_{int}(t'') \dots \hat{H}_{int}(t^{(n-1)})
\end{equation}

\noindent By the triangle inequality and the product inequality, the norm of the $n^{th}$ term has an upper bound
\begin{equation}
|| \int \dots ||_{\infty} \le \frac{(t-t_0)^n}{n!} (|| \hat{H}_{int} ||_\infty^*)^n .
\end{equation}

\noindent And, since $\hat{H}_{int}(t)$ is just unitarily rotated from $\hat{H}_{int}$
\begin{equation}
|| \hat{H}_{int} ||_\infty^* = \sup_{t' \in [t,t_0]} || \hat{H}_{int}(t') ||_\infty = (||\hat{H}_{int}||_\infty)_S .
\end{equation}

\noindent If $||\hat{H}_{int}||_\infty << 1$, the series has a nonzero radius of convergence. \\

\noindent \textbf{Theorem:} $\hat{U}(t,t_0) = \mathcal{T}[ e^{-i \int^t_{t_0} dt' \, \hat{H}_{int}(t')}]$, where $\mathcal{T}[\,]$ is the time-ordering operator. \\

\noindent To prove, expand the right-hand side in a Taylor series, apply time-ordering, and check that the two sides are equal. \\

\subsection*{Observables in QFT}

\noindent An important observable in QFT is scattering cross sections in scattering experiments. Namely, the $\mathcal{S}$-matrix is determined by Green's functions ($n$-point correlation functions)
\begin{equation}
G^{(n)}(x_1,x_2,\dots,x_n) = \bra{\Omega} \mathcal{T} [ \hat{\phi}_{1H} \hat{\phi}_{2H} \dots \hat{\phi}_{nH} ] \ket{\Omega}
\end{equation}

\noindent Where $\ket{\Omega}$ is the vacuum state of the Hamiltonian, and the subscript "$H$" denotes the Heisenberg picture, such that $\hat{\phi}_{jH} = \hat{\phi}(\textbf{x}_j) = \hat{\phi}(t_j,x_j)$. \\

\noindent \textbf{Claim:} 
\begin{equation}
G^{(n)}(x_1,x_2,\dots,x_n) = \frac{ \bra{0} \mathcal{T}[ \hat{\phi}_{1I} \hat{\phi}_{2I} \dots \hat{\phi}_{nI}\hat{\mathcal{S}}] \ket{0}}{\bra{0} \hat{\mathcal{S}} \ket{0}}
\end{equation}

Where $\bra{\phi}\hat{\mathcal{S}}\ket{\psi} = \lim_{t_\pm \rightarrow \pm \infty} \bra{\phi} \hat{U}(t_+, t_-) \ket{\psi}$, and $\hat{H}_0\ket{0} = 0$. \\

\noindent \textbf{Proof:} \\
\noindent Assume that $t_1>t_2>\dots>t_n$. \\
\noindent Then the right-hand side of the numerator reads
\begin{align}
&\bra{0} \hat{U}(\infty,t_1) \hat{\phi}_{1I} \hat{U}(t_1,t_2) \hat{\phi}_{2I} \dots \hat{\phi}_{nI} \hat{U}(t_n,-\infty) \ket{0} \\
&= \bra{0} \hat{U}(\infty,t_1) \hat{\phi}_{1H} \hat{\phi}_{2H} \dots \hat{\phi}_{nH} \hat{U}(t_0,-\infty) \ket{0}
\end{align}
\noindent Where $\hat{\phi}_H(t,x) = \hat{U}^\dagger(t,t_0) \hat{\phi}_I(t,x) \hat{U}(t,t_0)$ \\
\noindent Now dealing with 
\begin{align}
\hat{U}(t_0,-\infty) \ket{0} &= \lim_{t'\to -\infty} \lim_{t \to t_0} e^{i \hat{H}_0(t-t_0)} e^{-i \hat{H}(t-t')} e^{-i \hat{H}_0(t'-t)} \ket{0} \\
&= \lim_{t' \to -\infty} (\ket{\Omega}\bra{\Omega} + \sum_{n>0} e^{-i E_n (t_0 - t')} \ket{E_n}\bra{E_n})\ket{0}
\end{align}
\noindent Where the nonvanishing terms are written in the eigenbasis of $\hat{H}$. \\
\noindent Quantum fields have a continuous spectra, such that $\sum_{n>0} \sim \int dE$. \\
\noindent Invoke the Riemann-Lebesgue Lemma ($\lim_{k\to\infty} \hat{\psi}(k) = \lim_{k\to\infty} \int \psi(x)e^{ikx} dx = 0$), and consider
\begin{equation}
\lim_{t' \to -\infty} \int dE \braket{E|0} e^{-iE(t-t_0)} \braket{\phi|E} = 0
\end{equation}
\noindent The numerator is now equal to 
\begin{equation}
\braket{0|\Omega}\braket{\Omega|0}\bra{\Omega}\hat{\phi}_{1H}\dots\hat{\phi}_{nH}\ket{\Omega}
\end{equation}
\noindent Where $\braket{0|\Omega}\braket{\Omega|0}$ is equal to the denominator and cancels. QED.\\

\noindent Essentially, interacting quantum field theories come down to throwing in a Taylor series for the $\mathcal{S}$-matrix and $\hat{\phi}_{iI}$, and truncating some terms.